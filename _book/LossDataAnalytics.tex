\documentclass[]{book}
\usepackage{lmodern}
\usepackage{amssymb,amsmath}
\usepackage{ifxetex,ifluatex}
\usepackage{fixltx2e} % provides \textsubscript
\ifnum 0\ifxetex 1\fi\ifluatex 1\fi=0 % if pdftex
  \usepackage[T1]{fontenc}
  \usepackage[utf8]{inputenc}
\else % if luatex or xelatex
  \ifxetex
    \usepackage{mathspec}
  \else
    \usepackage{fontspec}
  \fi
  \defaultfontfeatures{Ligatures=TeX,Scale=MatchLowercase}
\fi
% use upquote if available, for straight quotes in verbatim environments
\IfFileExists{upquote.sty}{\usepackage{upquote}}{}
% use microtype if available
\IfFileExists{microtype.sty}{%
\usepackage{microtype}
\UseMicrotypeSet[protrusion]{basicmath} % disable protrusion for tt fonts
}{}
\usepackage[margin=1in]{geometry}
\usepackage{hyperref}
\hypersetup{unicode=true,
            pdftitle={Loss Data Analytics},
            pdfauthor={An open text authored by the Actuarial Community},
            pdfborder={0 0 0},
            breaklinks=true}
\urlstyle{same}  % don't use monospace font for urls
\usepackage{natbib}
\bibliographystyle{apalike}
\usepackage{longtable,booktabs}
\usepackage{graphicx,grffile}
\makeatletter
\def\maxwidth{\ifdim\Gin@nat@width>\linewidth\linewidth\else\Gin@nat@width\fi}
\def\maxheight{\ifdim\Gin@nat@height>\textheight\textheight\else\Gin@nat@height\fi}
\makeatother
% Scale images if necessary, so that they will not overflow the page
% margins by default, and it is still possible to overwrite the defaults
% using explicit options in \includegraphics[width, height, ...]{}
\setkeys{Gin}{width=\maxwidth,height=\maxheight,keepaspectratio}
\IfFileExists{parskip.sty}{%
\usepackage{parskip}
}{% else
\setlength{\parindent}{0pt}
\setlength{\parskip}{6pt plus 2pt minus 1pt}
}
\setlength{\emergencystretch}{3em}  % prevent overfull lines
\providecommand{\tightlist}{%
  \setlength{\itemsep}{0pt}\setlength{\parskip}{0pt}}
\setcounter{secnumdepth}{5}
% Redefines (sub)paragraphs to behave more like sections
\ifx\paragraph\undefined\else
\let\oldparagraph\paragraph
\renewcommand{\paragraph}[1]{\oldparagraph{#1}\mbox{}}
\fi
\ifx\subparagraph\undefined\else
\let\oldsubparagraph\subparagraph
\renewcommand{\subparagraph}[1]{\oldsubparagraph{#1}\mbox{}}
\fi

%%% Use protect on footnotes to avoid problems with footnotes in titles
\let\rmarkdownfootnote\footnote%
\def\footnote{\protect\rmarkdownfootnote}

%%% Change title format to be more compact
\usepackage{titling}

% Create subtitle command for use in maketitle
\newcommand{\subtitle}[1]{
  \posttitle{
    \begin{center}\large#1\end{center}
    }
}

\setlength{\droptitle}{-2em}
  \title{Loss Data Analytics}
  \pretitle{\vspace{\droptitle}\centering\huge}
  \posttitle{\par}
  \author{An open text authored by the Actuarial Community}
  \preauthor{\centering\large\emph}
  \postauthor{\par}
  \predate{\centering\large\emph}
  \postdate{\par}
  \date{2018-05-21}

\usepackage{booktabs}
\setcounter{secnumdepth}{2}

\usepackage{amsthm}
\newtheorem{theorem}{Theorem}[chapter]
\newtheorem{lemma}{Lemma}[chapter]
\theoremstyle{definition}
\newtheorem{definition}{Definition}[chapter]
\newtheorem{corollary}{Corollary}[chapter]
\newtheorem{proposition}{Proposition}[chapter]
\theoremstyle{definition}
\newtheorem{example}{Example}[chapter]
\theoremstyle{definition}
\newtheorem{exercise}{Exercise}[chapter]
\theoremstyle{remark}
\newtheorem*{remark}{Remark}
\newtheorem*{solution}{Solution}
\begin{document}
\maketitle

{
\setcounter{tocdepth}{2}
\tableofcontents
}
\chapter{Referencing}\label{referencing}

This is a site to help with referencing within bookdown.

\chapter{Aggregate Loss Models TEX
TEST}\label{aggregate-loss-models-tex-test}

Placeholder

\section{Referencing a section}\label{referencing-a-section}

\section{Equations}\label{S:Referencing-equations}

\section{Tables}\label{S:Referencing-tables}

\subsection{Using equation style referencing for handmade converted
latex
table}\label{using-equation-style-referencing-for-handmade-converted-latex-table}

\subsection{Using equation style referencing for original latex
table}\label{using-equation-style-referencing-for-original-latex-table}

\subsection{Using table style on R generated
tables}\label{using-table-style-on-r-generated-tables}

\subsection{Using converted tex and in-block
referencing}\label{using-converted-tex-and-in-block-referencing}

\subsection{Using original Latex code and in-block
referencing}\label{using-original-latex-code-and-in-block-referencing}

\section{Figures}\label{S:Referencing-figures}

\chapter{Risk classification}\label{risk-classification}

Placeholder

\section{Introduction}\label{introduction}

\section{Poisson regression model}\label{poisson-regression-model}

\subsection{Need of Poisson regression}\label{S:Need.Poi.reg}

\subsection{Poisson regression}\label{poisson-regression}

\subsection{Incorporating exposure}\label{incorporating-exposure}

\subsection{Exercises}\label{exercises}

\section{Categorical variables and multiplicative
tariff}\label{categorical-variables-and-multiplicative-tariff}

\subsection{Rating factors and tariff}\label{rating-factors-and-tariff}

\section{Multiplicative tariff model}\label{multiplicative-tariff-model}

\subsection{Poisson regression for multiplicative
tariff}\label{poisson-regression-for-multiplicative-tariff}

\subsection{Numerical examples}\label{numerical-examples}

\section{Further Reading and
References}\label{further-reading-and-references}

\section{Technical supplements -- Estimating Poisson regression
model}\label{S:mle.Pois.reg}

\bibliography{Bibliography/LDAReference.bib,Bibliography/LDAReferenceChap4.bib,Bibliography/dsref.bib,Bibliography/packages.bib}


\end{document}
