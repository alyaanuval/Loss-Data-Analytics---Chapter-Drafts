\documentclass[12pt,letterpaper]{article}
\usepackage{url}
\usepackage{graphicx}
\usepackage{amssymb}
\usepackage{makeidx}
\usepackage{amsmath}
\usepackage{eurosym}
\usepackage{amsfonts}
\usepackage{rotating}
\usepackage{scalefnt}
\usepackage{float}
\usepackage{geometry}
%\geometry{left=0.6in,right=0.6in,top=0.6in,bottom=0.6in}
\geometry{left=0.8in,right=0.8in,top=1in,bottom=1in}
\usepackage{alltt}
\usepackage{multicol}
\usepackage{setspace}
\usepackage{lscape}
\usepackage{longtable}
\usepackage{color}
\begin{document}

\title{Aggregate Loss Models}

\author{
    Peng Shi \\
    Wisconsin School of Business\\
    University of Wisconsin--Madison\\
    Email: pshi@bus.wisc.edu\\
    \and
    Lisa Gao \\
    Wisconsin School of Business\\
    University of Wisconsin--Madison\\
    Email: lisa.gao@wisc.edu\\
}
\maketitle

\abstract{
This chapter introduces probability models for describing the aggregate claims that arise from a portfolio of insurance contracts. We presents two standard modeling approaches, the individual risk model and the collective risk model. Further, we discuss strategies for computing the distribution of the aggregate claims. Finally, we examine the effects of individual policy modifications on the aggregate loss distribution.

}

\scalefont{0.8} \tableofcontents

\scalefont{1.25}


\section{Introduction}

The objective of this chapter is to build a probability model to describe the aggregate claims by an insurance system occurring in a fixed time period. The insurance system could be a single policy, a group insurance contract, a business line, or an entire book of an insurance's business. In the chapter, aggregate claims refers to either the number or the amount of claims from a portfolio of insurance contracts. However, the modeling framework is readily to apply in the more general setup.

Consider an insurance portfolio of $n$ individual contracts, and let $S$ denote the aggregate losses of the portfolio in a given time period. There are two approaches to modeling the aggregate losses $S$, the individual risk model and the collective risk model. The individual risk model emphasizes the loss from each individual contract and represents the aggregate losses as:
\begin{align*}
S=X_1 +X_2 +\cdots+X_n,
\end{align*}
where $X_i~(i=1,\ldots,n)$ is interpreted as the loss amount from the $i$th contract. It is worth stressing that $n$ denotes the number of contracts in the portfolio and thus is a fixed number rather than a random variable. For the individual risk model, one usually assumes $X_{i}$'s are independent, i.e., $X_{i}\perp X_{j}$ $\forall $ $i,j$. Because of different contract features such as coverage and exposure, $X_{i}$'s are not necessarily identically
distributed. A notable feature of the distribution of each $X_i$ is the probability mass at zero corresponding to the event of no claims.

The collective risk model represents the aggregate losses in terms of a frequency distribution and a severity distribution:
\begin{align*}
S=X_1 +X_2 +\cdots+X_N.
\end{align*}
Here one thinks of a random number of claims $N$ that may represent either the number of losses or the number of payments. In contrast, in the individual risk model, we use a fixed number of contracts $n$. We think of $X_1, X_2, \ldots, X_N$ as representing the amount of each loss. Each loss may or may not corresponding to a unique contract. For instance, there may be multiple claims arising from a single contract. It is natural to think about $X_i>0$ because if $X_i=0$ then no claim has occurred. Typically we assume that conditional on $N=n$, $X_{1},X_{2},\cdots ,X_{n}$ are i.i.d. random variables. The distribution of $N$ is known as the frequency distribution, and the common distribution of $X$ is known as the severity distribution. We further assume $N$ and $X$ are independent. With the collective risk model, we may decompose the aggregate losses into the frequency ($N$) process and the severity ($X$) process. This flexibility allows the analyst to comment on these two separate components. For example, sales growth due to lower underwriting standards could lead to higher frequency of losses but might not affect severity. Similarly, inflation or other economic forces could have an impact on severity but not on frequency.


\section{Individual Risk Model}

As noted earlier, for the individual risk model, we think of $X_i$ as the loss from $i^{th}$ contract and interpret
\begin{eqnarray*}
S_n=X_1 +X_2 +\cdots+X_n
\end{eqnarray*}
to be the aggregate loss from all contracts in a portfolio or group of contracts. Under the independence assumption on $X_i's$, it is straightforward to show
 \begin{align*}
    {\rm E}(S_n) &= \sum_{i=1}^{n} {\rm E}(X_i),~~~~
    {\rm Var}(S_n) = \sum_{i=1}^{n} {\rm Var}(X_i)\\
    P_{S_n}(z) &= \prod_{i=1}^{n}P_{X_i}(z), ~~~~
    M_{S_n}(t) = \prod_{i=1}^{n}M_{X_i}(t) \\
 \end{align*}
where $P_S(\cdot)$ and $M_S(\cdot)$ are probability generating function and moment generating function of $S$, respectively. The distribution of each $X_i$ contains mass at zero, corresponding to the event of no claim. One strategy to incorporate the zero mass in the distribution is using the two-part framework:
\begin{align*}
X_i = I_i\times B_i = \left\{\begin{array}{ll}
                               0 & I_i=0 \\
                               B_i & I_i=1
                             \end{array}
             \right.
\end{align*}
Here $I_i$ is a Bernoulli variable indicating whether or not a loss occurs for the $i$th contract, and $B_i$, a r.v. with nonnegative support, represents the amount of losses of the contract given loss occurrence. Assume that $I_1 ,\ldots,I_n ,B_1 ,\ldots,B_n$ are mutually independent. Denote ${\rm Pr} (I_i =1)=q_i$, $\mu_i={\rm E}(B_i)$, and $\sigma_i^2={\rm Var}(B_i)$. One can show
\begin{align*}
\mathrm{E}(S_n)& =\sum_{i=1}^n ~q_i  ~\mu _j \\
\mathrm{Var}(S_n) & =\sum_{i=1}^n \left( q_i \sigma _i^2+q_i (1-q_j)\mu_i^2 \right)\\
P_{S_n}(z) & =\sum_{i=1}^n \left( 1-q_i+q_i P_{B_i}(z) \right)\\
M_{S_n}(t) & =\sum_{i=1}^n \left( 1-q_i+q_i M_{B_i}(t) \right)
\end{align*}
A special case of the above model is when $B_i$ follows a degenerate distribution with $\mu_i=b_i$ and $\sigma^2_i=0$. One example is term life insurance or a pure endowment insurance where $b_i$ represents the amount of insurance of the $i$th contract.

Another strategy to accommodate zero mass in the distribution of $X_i$ is a collective risk model, i.e. $X_i=Z_{i1}+\cdots+Z_{iN_i}$ where $X_i=0$ when $N_i=0$. The collective risk model will be discussed in detail in the next section. \\


%{\color{red} Lisa clean up the example below}\\

\noindent \textit{Example. Course 3, May 2000, 19.} An insurance company sold 300 fire insurance policies as follows:
\bigskip
\begin{center}
	\begin{tabular}{c c c} \hline
		Number of & Policy & Probability of\\
		Policies &  Maximum &  Claim Per Policy\\ \hline
		100 & 400 & 0.05\\
		200 & 300 & 0.06\\ \hline
	\end{tabular}
\end{center}
\bigskip
You are given:
\begin{enumerate}
	\item[(i)] The claim amount for each policy is uniformly distributed between $0$ and the policy maximum.
	\item[(ii)] The probability of more than one claim per policy is $0$.
	\item[(iii)] Claim occurrences are independent.
\end{enumerate}

\noindent Calculate the mean $\mathrm{E~}S_n$ and variance $\mathrm{Var~}S_n$ of the aggregate claims. How would these results change if every claim is equal to the policy maximum?

\bigskip

\noindent \textit{Solution.} The aggregate claims are $S_{300} = X_1+\cdots
+X_{300}$. Policy claims amounts are uniformly distributed on $(0,
PolMax)$, so the mean claim amount is $PolMax/2$ and the variance is
$PolMax^2/12$. Thus, for policy $i=1,...,300$, we have
\begin{center}
	\begin{tabular}{c c ccc} \hline
		Number of & Policy & Probability of & Mean& Variance\\
		Policies &  Maximum &  Claim Per Policy& Amount & Amount\\
		& & ($q_i$) & ($\mu_i$) & ($\sigma_i^2$) \\ \hline
		100 & 400 & 0.05 & 200 & $400^2/12$\\
		200 & 300 & 0.06 & 150 & $300^2/12$ \\ \hline
	\end{tabular}
\end{center}

\noindent The mean of the aggregate claims is
$$\mathrm{E~} S_{300} = \sum_{i=1}^{300} q_i \mu_i = 100\left\{0.05(200)\right\} + 200\left\{0.06 (150) \right\} = 1,000+1,824 =
2,824$$

\noindent The variance of the aggregate claims is
\begin{eqnarray*}
	\mathrm{Var~}S_{300} &=& \sum_{i=1}^{300} \left( q_i \sigma _i^2+q_i (1-q_i
	)\mu_i^2 \right) \\
	&=& 100\left\{ 0.05 \left(\frac{400^2}{12}\right) +0.05 (1-0.05 )200^2 \right\}+
	200\left\{
	0.06 \left(\frac{300^2}{12}\right) +0.06 (1-0.06 )150^2 \right\}\\
	&=& 600,466.67 .
\end{eqnarray*}
\begin{flushright}$\Box$\end{flushright}

\noindent \textit{Follow-Up.} Now suppose everybody receives the policy maximum if a claim occurs. What is the expected aggregate loss and variance of the aggregate loss? Each policy claim amount $B_i$ is now fixed at $PolMax$ instead of random, so $\sigma_i^2 = \mathrm{Var~} B_i = 0$ and $\mu_i = PolMax$.
\begin{align*}
\mathrm{E~}S^X &= \sum_{i=1}^{300} q_i \mu_i = 100 \left\{0.05(400) \right\} + 200 \left\{ 0.06(300) \right\} = 5,648
\end{align*}
\begin{align*}
\mathrm{Var~}S^X &= \sum_{i=1}^{300} \left( q_i \sigma _i^2+q_i (1-q_i
)\mu_i^2 \right) = \sum_{i=1}^{300} \left( q_i (1-q_i) \mu_i^2 \right) \\
&= 100 \left\{(0.05) (1-0.05) 400^2\right\} +
200 \left\{(0.06) (1-0.06)300^2\right\} \\
&= 76,000 + 101,520 = 177,520
\end{align*}
\begin{flushright}$\Box$\end{flushright}

The individual risk model can also be used for claim frequency. If $X_i$ denotes the number of claims from the $i$th contract, and $S_n$ is interpreted as the total number of claims from the portfolio. In this case, the above two-part framework still applies. Assume $X_i$ belongs to the $(a,b,0)$ class with pmf denoted by $p_{ik}$. Let $X_i^{T}$ denote the associated zero-truncated distribution in the $(a,b,1)$ class with the pmf $p_{ik}^T=p_{ik}/(1-p_{i0})$ for $k=1,2,\ldots$. Using the relationship between their generating functions:
\begin{align*}
P_{X_i}(z) = p_{i0} +(1-p_{i0}) P_{X_i^{T}}(z),
\end{align*}
we can write $X_i=I_i\times B_i$ with $q_i={\rm Pr}(I_i=1)={\rm Pr}(X_i>0)=1-p_{i0}$ and $B_i=X_i^T$.\\


%{\color{red} maybe Lisa give an example below}\\

\noindent \textit{Example.} An insurance company sold a portfolio of 100 independent homeowners insurance policies, each of which has claim frequency following a zero-modified Poisson distribution, as follows:

\begin{center}
	\begin{tabular}{c c cc} \hline
		Type of & Number of  & Probability of & $\lambda$ \\
		Policy & Policies  &  At Least 1 Claim&  \\ \hline
		Low-risk & 40 & 0.03 & 1 \\
		High-risk & 60 & 0.05 & 2 \\ \hline
	\end{tabular}
\end{center}
Find the expected value and variance of the claim frequency for the entire portfolio.

\bigskip

\noindent \textit{Solution.}
For each policy, we can write the zero-modified Poisson claim frequency $N_i$ as $N_i = I_i \times B_i$, where
$$q_i = \Pr(I_i = 1) = \Pr(N_i > 0) = 1-p_{i0}$$
For the low-risk policies, we have $q_i = 0.03$ and for the high-risk policies, we have $q_i=0.05$. Further, $B_i = N_i^T$, the zero-truncated version of $N_i$. Thus, we have
\begin{align*}
\mu_i &={\rm E}(B_i) = {\rm E}(N_i^T) = \frac{\lambda}{1-e^{-\lambda}} \\
\sigma_i^2 &={\rm Var}(B_i) = {\rm Var}(N_i^T) = \frac{\lambda [1-(\lambda+1)e^{-\lambda}]}{(1-e^{-\lambda})^2}
\end{align*}
Let the portfolio claim frequency be $S_n = \sum_{i=1}^n N_i$. Using the formulas above, the expected claim frequency of the portfolio is
\begin{align*}
	\mathrm{E~} S_n &= \sum_{i=1}^{100} q_i \mu_i \\
	& = 40\left[0.03 \left(\frac{1}{1-e^{-1}} \right) \right] + 60 \left[0.05 \left( \frac{2}{1-e^{-2}} \right) \right] \\
	&= 40(0.03)(1.5820) + 60(0.05)(2.3130) = 8.8375
\end{align*}
The variance of the claim frequency of the portfolio is
\begin{align*}
	\mathrm{Var~}S_n &= \sum_{i=1}^{100} \left( q_i \sigma _i^2+q_i (1-q_i
	)\mu_i^2 \right) \\
	&= 40 \left[ 0.03 \left(\frac{1-2e^{-1}}{(1-e^{-1})^2} \right) + 0.03(0.97)(1.5820^2) \right] + 60 \left[0.05 \left( \frac{2[1-3e^{-2}]}{ (1-e^{-2})^2} \right) + 0.05(0.95)(2.3130^2) \right] \\
	&= 23.7214
\end{align*}
Note that equivalently, we could have calculated the mean and variance of an individual policy directly using the relationship between the zero-modified and zero-truncated Poisson distributions.

\begin{flushright}$\Box$\end{flushright}

To understand the distribution of the aggregate loss, one could use central limit theorem to approximate the distribution of $S_n$. Denote $\mu_S={\rm E}(S)$ and $\sigma^2_S={\rm Var}(S)$, the cdf of $S_n$ is:
\begin{align*}
 F_{S_n}(s)={\rm Pr}({S_n}\leq s) = \Phi \left(\frac{s-\mu_S}{\sigma_S}\right).
\end{align*}


%{\color{red} maybe Lisa give an example below}\\
\noindent \textit{Example. Course 3, May 2000, 19 (Follow-Up).} As in the original example earlier, an insurance company sold 300 fire insurance policies, with claim amounts uniformly distributed between 0 and the policy maximum. Using the normal approximation, calculate the probability that the aggregate claim amount exceeds \$3,500.

\bigskip

\noindent \textit{Solution.}
We have seen earlier that $\mathrm{E~} S_{300}=2,824$ and $\mathrm{Var~}S_{300} = 600,466.67$. Then
\begin{align*}
{\rm Pr}(S_{300} > 3,500) &= 1 - {\rm Pr}(S_{300} \leq 3,500) \\
&= 1- \Phi \left( \frac{3,500-2,824}{\sqrt{600,466.67}} \right) = 1 - \Phi \left( 0.87237 \right) \\
&= 1 - 0.8085 = 0.1915
\end{align*}
\begin{flushright}$\Box$\end{flushright}

For small $n$, the distribution of $S_n$ is likely skewed, and the normal approximation would be a poor choice. To examine the aggregate loss distribution, we go back to the basics and first principles. Specifically, the distribution can be derived recursively. Define $S_k=X_1 + \cdots + X_k, k=1,\ldots,n$, we have:
For $k=1$:
$$F_{S_1}(s) = {\rm Pr}(S_1\leq s) = {\rm Pr}(X_1\leq s) = F_{X_1}(s).$$
For $k=2,\ldots,n$,
   \begin{align*}
    F_{S_k}(s)&={\Pr}(X_1+\cdots+X_k\leq s) ={\Pr}(S_{k-1}+X_k\leq s) \\
    &={\rm E}_{X_k}\left[{\rm Pr}(S_{k-1}\leq s-X_k|X_k)\right]= {\rm E}_{X_k}\left[F_{S_{k-1}}(s-X_k)\right].
   \end{align*}

There are some simple cases where the $S_n$ has a closed form. Examples include
  \begin{itemize}
   \item If $X_i\sim N(\mu_i,\sigma_i^2)$, then $S_n\sim N(\sum_{i=1}^{n}\mu_i,\sum_{i=1}^{n}\sigma_i^2)$
   \item If $X_i\sim Exponential(\theta)$, then $S_n\sim Gamma(n,\theta)$
   \item If $X_i\sim Gamma(\alpha_i,\theta)$, then $S_n\sim Gamma(\sum_{i=1}^n\alpha_i,\theta)$
   \item If $X_i\sim Poisson(\lambda_i)$, then $S_n\sim Poisson(\sum_{i=1}^{n}\lambda_i)$
   \item If $X_i\sim Bin(q,m_i)$, then $S_n\sim Bin(q,\sum_{i=1}^n m_i)$
   \item If $X_i\sim Geometric(\beta)$, then $S_n\sim NegBin(\beta,n)$
   \item If $X_i\sim NegBin(\beta,r_i)$, then $S_n\sim NegBin(\beta,\sum_{i=1}^n r_i)$
  \end{itemize}



A special case is when $X_i's$ are identically distributed. Let $F_X(x)={\Pr}(X\leq x)$ be the common distribution of $X_i$ $(i=1,\ldots,n)$, we define
   $$F^{*n}_X(x)={\Pr}(X_1+\cdots+X_n\leq x)$$
the $n$-fold convolution of $F_X$.

\bigskip

\noindent \textit{Example: Gamma Distribution}. For an easy case, assume that $X_i \sim$ gamma with parameters $(\alpha, \theta) $. As we know, the moment generating function (mgf) is $M_{X}(t) = (1 - \theta t)^{- \alpha}$. Thus, the mgf of the sum $S_n = X_1 + \cdots + X_n$ is
\begin{eqnarray*}
M_{S_n}(t) = \mathrm{E~} \exp(t(X_1 + \cdots + X_n)) = (1 - \theta t)^{-n \alpha} ,
\end{eqnarray*}
Thus, $S_n$ has a gamma distribution with parameters $(n \alpha, \theta) $. This makes it easy to compute $F^{\ast n}(x) = \Pr(S_n \le x).$ This property is known as ``closed under convolution''.
\begin{flushright}$\Box$\end{flushright}


%{\color{red} Lisa give another example on NegBin using probability generating function}\\
\noindent \textit{Example: Negative Binomial Distribution.} Assume $X_i \sim NegBin(\beta, r_i)$. The probability generating function (pgf) is $P_X(z) = \left[1-\beta(z-1) \right]^{-r}$. Thus, the pgf of the sum $S_n =X_1+\cdots+X_n$ is
\begin{align*}
P_{S_n}(z) &= \mathrm{E~}\left[ z^{S_n} \right] = \mathrm{E~}\left[ z^{X_1+\cdots+X_n} \right] = \mathrm{E~}\left[ z^{X_1} z^{X_2} \cdots z^{X_n} \right] = \mathrm{E~}\left[z^{X_1}\right] \cdots \mathrm{E~}\left[z^{X_n}\right] \\
&= \prod_{i=1}^n P_{X_i}(z) = \prod_{i=1}^n \left[1-\beta(z-1) \right]^{-r_i} = \left[1-\beta(z-1) \right]^{-\sum_{i=1}^n r_i}
\end{align*}
Thus, $S_n$ has a negative binomial distribution with parameters $(\beta, \sum_{i=1}^n r_i)$.
\begin{flushright}$\Box$\end{flushright}


More generally, we can compute $F^{\ast n}$ recursively. Begin the recursion at $n=1$ using $F^{\ast 1} \left(x \right) =F(x)$. Next, for $n=2$, we have
\begin{eqnarray*}
F^{\ast 2} \left(x \right) &=& \Pr(X_1 + X_2 \le x) = \mathrm{E}_{X_2} \Pr(X_1 \le x - X_2|X_2)\\
&=& \mathrm{E}_X F(x - X)\\
&=& \int_{0}^{x} F(x-y) f(y) dy \:\:\:\:\:\:\:\:\:\:\:\:\:\:\text{continuous}\\
&=& \sum_{y \le x} F(x-y) f(y) \:\:\:\:\:\:\:\:\:\:\:\:\:\:\:\:\:\:\;\text{discrete}\\
\end {eqnarray*}

Recall $F(0) = 0$.
\bigskip
%Page 10 of notes

Similarly, let $S_n = X_1 + X_2 + \cdots + X_n$
\begin{eqnarray*}
F^{\ast n}\left(x\right) &=& \Pr(S_n \le x) = \Pr(S_{n-1} + X_n \le x)\\
&=&\mathrm{E}_{X_n}\Pr(S_{n-1} \le x - X_n|X_n)\\
&=&\mathrm{E}_X F^{\ast(n-1)}(x - X)\\
&=& \int_{0}^{x} F^{\ast(n-1)}(x-y)f(y)dy \:\:\:\text{continuous}\\
&=& \sum_{y \le x} F^{\ast(n-1)}(x-y)f(y) \:\:\:\:\:\:\:\:\text{discrete}\\
\end{eqnarray*}
\bigskip


%{\color{red} Lisa provide an example below}\\

\noindent \textit{Example. SOA Sample Question, 283 (modified).} The annual number of doctor visits for each individual in a family of 4 has geometric distribution with mean 1.5. The annual numbers of visits for the family members are mutually independent. An insurance pays 100 per doctor visit beginning with the 4th visit per family. Calculate the probability that the family will receive an insurance payment this year.

\bigskip

\noindent \textit{Solution.} Let $X_i \sim Geometric(\beta=1.5)$ be the number of doctor visits for one individual in the family and $S_4 = X_1 + X_2 + X_3 + X_4$ be the number of doctor visits for the family. The sum of 4 independent geometric distributions each with mean 1.5 follows a negative binomial distribution, i.e. $S_4 \sim NegBin(\beta=1.5, r=4)$.

\bigskip

\noindent If the insurance pays 100 per visit beginning with the 4th visit for the family, then the family will not receive an insurance payment if they have less than 4 claims. This probability is
\begin{align*}
	\Pr(S_4 < 4) &= \Pr(S_4 = 0) + \Pr(S_4 = 1) + \Pr(S_4 = 2) +\Pr(S_4 = 3) \\
	&= (1+1.5)^{-4} + \frac{4(1.5)}{(1+1.5)^5} + \frac{4(5)(1.5^2)}{2(1+1.5)^6} + \frac{4(5)(6)(1.5^3)}{3!(1+1.5)^7}\\
	&= 0.0256 + 0.0614 + 0.0922 + 0.1106 = 0.2898
\end{align*}
\begin{flushright}$\Box$\end{flushright}

\section{Collective Risk Model}
\subsection{Moments and Distribution}
Under the collective risk model $S=X_1+\cdots+X_N$, $\{X_i\}$ are i.i.d., and independent of $N$. Let $\mu = {\rm E}\left( X_{i}\right)$ and $\sigma ^{2} = {\rm Var}\left(X_{i}\right)$ $\forall $ $i$. Using the law of iterated expectations, the mean is
\begin{eqnarray*}
{\rm E}(S)={\rm E}_N[{\rm E}_S(S|N)] = {\rm E}_N[N\mu] = \mu {\rm E}(N).
\end{eqnarray*}
Using the law of total variation, the variance is
\begin{align*}
{\rm Var}(S) &= {\rm E}_N[{\rm Var}_S(S|N)] + {\rm Var}_N[{\rm E}_S(S|N)] \\
&={\rm E}_N[N\sigma^2] + {\rm Var}_N[N\mu] \\
&=\sigma^2{\rm E}[N] + \mu^2{\rm Var}[N]
\end{align*}


%{\color{red} Lisa clean up the example below}\\


\noindent \textit{Special Case: Poisson Distributed Frequency.} If $N \sim Poisson (\lambda)$, then
\begin{eqnarray*}
\mathrm{E~}N &=& \mathrm{Var~}N = \lambda\\
\mathrm{Var~}S &=& \lambda (\sigma^2 + \mu^2) = \lambda ~\mathrm{E~} X^2 .
\end{eqnarray*}

%Page 3 of notes
\noindent \textit{Example. Course 3, May 2001, 36.} The number of accidents follows a Poisson distribution with mean 12. Each accident generates 1, 2, or 3 claimants with probabilities 1/2, 1/3, and 1/6 respectively.

Calculate the variance in the total number of claimants.

\bigskip

\noindent \textit{Solution:}
$$\mathrm{E~}X^2 = 1^2 \left( \frac{1}{2}\right) + 2^2\left(\frac{1}{3} \right) + 3^2\left(\frac{1}{6}\right)
= \frac{10}{3}$$
$$\mathrm{Var~}S = \lambda \ \mathrm{E~}X^2 = 12\left(\frac{10}{3}\right) = 40$$

\noindent Alternatively, using the general approach, $\mathrm{Var~}S = \sigma^2 \mathrm{E~}N + \mu^2 \mathrm{Var~}N$, where

$$\mathrm{E~}N = \mathrm{Var~}N = 12$$
$$\mu = \mathrm{E~}X = 1\left(\frac{1}{2}\right) + 2\left(\frac{1}{3}\right) + 3\left(\frac{1}{6}\right)
= \frac{5}{3}$$
$$\sigma^2 = \mathrm{E~}X^2 - (\mathrm{E~}X)^2 = \frac{10}{3} - \frac{25}{9}
= \frac{5}{9}$$
$$\Rightarrow \ \mathrm{Var~}S = \left(\frac{5}{9}\right)\left(12\right) + \left(\frac{5}{3}\right)^2\left(12\right) = 40$$
\begin{flushright}$\Box$\end{flushright}
%Page 4 of notes


In general, the moments of $S$ can be derived from its moment generating function (mgf). Because $\{X_i\}$ are i.i.d., we denote the mgf of $X$ as $M_{X}(t) = \mathrm{E~}(e^{tX})$. Using the law of iterated expectations, the mgf of $S$ is
\begin{eqnarray*}
M_{S}(t) &=& \mathrm{E}(e^{St})=\mathrm{E}[\mathrm{E}(e^{St}|N)]\\
&=& \mathrm{E~}[(M_{X}(t))^N]
\end{eqnarray*}
where we use the relation $\mathrm{E}[e^{t(X_1+\cdots+X_n)}] = \mathrm{E}(e^{tX_1})\cdots\mathrm{E}(e^{tX_n})
= (M_{X}(t))^n$. Now, recall that the probability generating function (pgf) of $N$ is $P(z) = \mathrm{E}(z^N)$. Denote $M_{X}(t)=z$, it is shown
\begin{eqnarray*}
M_{S}(t) = \mathrm{E~}(z^N)  = P_{N}(z) = P_{N}[M_{X}(t)].
\end{eqnarray*}
Similarly, if $S$ a discrete, one can show the pgf of $S$ is:
\begin{eqnarray*}
P_{S}(z) = P_{N}[P_{X}(z)].
\end{eqnarray*}

To get $\mathrm{E~}S = M_{S}'(0)$, we use the chain rule
\[
M_{S}'(t) = \frac{\partial}{\partial t} P_{N}(M_{X}(t)) = P_{N}'(M_{X}(t)) M_{X}'(t)\\
\]
and recall $M_{X}(0) = 1, M_{X}'(0) = \mathrm{E~}X = \mu, P_{N}'(1) = \mathrm{E~}N$. So,
\begin{eqnarray*}
\mathrm{E~}S = M_{S}'(0) = P_{N}'(M_{X}'(0)) M_{X}'(0) = \mu {\rm E}(X)
\end{eqnarray*}
Similarly, one could use relation
\[
\mathrm{E~}S^2 = M_{S}''(0)
\]
to get
\[\mathrm{Var~}S = \sigma^2 \mathrm{E~}N + \mu^2 \mathrm{Var~}N.\]

\bigskip

\noindent \textit{Special Case. Poisson Frequency.} Let $N \sim Poisson (\lambda)$. Thus, the pgf of $N$ is $P_N (z) =\exp[\lambda(z-1)]$, and the mgf of $S$ is
\begin{eqnarray*}
M_{S}(t) &=&\exp[\lambda(M_{X}(t) - 1)].
\end{eqnarray*}
Taking derivatives yield
\begin{eqnarray*}
M_{S}(t) &=&\exp(\lambda(M_{X}(t) - 1))\\
M_{S}'(t) &=&\exp(\lambda(M_{X}(t) - 1)) \lambda M_{X}'(t)\\
&=& M_{S}(t) \lambda M_{X}'(t)\\
M_{S}''(t) &=& M_{S}(t) \lambda M_{X}''(t) + \{M_{S}(t) \lambda M_{X}'(t)\} \lambda M_{X}'(t)
\end{eqnarray*}
Evaluating these at $t=0$ yields
\begin{eqnarray*}
M_{S}''(0) &=& \lambda \mathrm{E}(X^2) + \lambda^2 \mu^2\\
\mathrm{Var~}S &=& \lambda \mathrm{E}(X^2) + \lambda^2 \mu^2 - (\lambda \mu)^2\\
&=& \lambda \mathrm{E}(X^2).
\end{eqnarray*}
\begin{flushright}$\Box$\end{flushright}


%{\color{red} Lisa clean up the example below}\\


%Page 7 of notes
\noindent \textit{Example. Course 3, May 2001, 29.} You are the producer of a television quiz show that gives cash prizes. The number of prizes, $N$, and prize amount, $X$, have the following distributions:\\

\begin{center}
\begin{tabular}{c c c c c}\hline
    $n$ & $\Pr(N=n)$ & & $x$ & $\Pr(X=x)$\\ \hline
    1 & 0.8 & & 0 & 0.2 \\
    2 & 0.2 & & 100 & 0.7 \\
       &       & & 1000 & 0.1\\\hline
  \end{tabular}
\end{center}
\bigskip

\noindent Your budget for prizes equals the expected aggregate cash prizes plus the standard deviation of aggregate cash prizes. Calculate your budget.

\bigskip

\noindent \textit{Solution.} We need to calculate the mean and standard deviation of the aggregate (sum) of cash prizes. The moments of the frequency distribution $N$ are
\begin{eqnarray*}
\mathrm{E~}N &=& 1 (0.8) + 2 (0.2) =1.2\\
\mathrm{E~}N^2 &=&  1^2 (0.8) + 2^2 (0.2) =1.6\\
\mathrm{Var~}N &=& \mathrm{E~}N^2 - \left( \mathrm{E~}N \right)^2= 0.16
\end{eqnarray*}
The moments of the severity distribution $X$ are
\begin{eqnarray*}
\mathrm{E~}X &=& 0 (0.2) + 100 (0.7) + 1000 (0.1) = 170 = \mu\\
\mathrm{E~}X^2 &=& 0^2 (0.2) + 100^2 (0.7) + 1000^2 (0.1) = 107,000\\
\mathrm{Var~}X &=& \mathrm{E~}X^2 - \left( \mathrm{E~}X \right)^2=78,100 = \sigma^2
\end{eqnarray*}
Thus, the mean and variance of the aggregate cash prize are
\begin{eqnarray*}
\mathrm{E~}S  &=& \mu \mathrm{E~}N = 170 (1.2) = 204 \\
\mathrm{Var~}S &=& \sigma^2 \mathrm{E~}N + \mu^2 \mathrm{Var~}N\\
&=& 78,100 (1.2) + 170^2 (0.16) = 98,344
\end{eqnarray*}
This gives the following required budget
\begin{eqnarray*}
Budget &=& \mathrm{E~}S + \sqrt{\mathrm{Var~}S} \\
&=& 204 + \sqrt{98,344} = 517.60 .
\end{eqnarray*}
\begin{flushright}$\Box$\end{flushright}
%Page 8 of notes


The distribution of $S$ is called a compound distribution, and it can be derived based on the convolution of $F_X$ as follows:
\begin{eqnarray*}
F_{S}(s) &=& \Pr \left(X_1 + \cdots + X_N \le s \right) \\
&=&  \mathrm{E} \left[ \Pr \left(X_1 + \cdots + X_N  \le s|N=n \right) \right]\\
&=&  \mathrm{E} \left[ F_{X}^{\ast N}(s) \right] \\
&=&  p_0 + \sum_{n=1}^{\infty }p_n F_{X}^{\ast n}(s)
\end{eqnarray*}


%{\color{red} Lisa clean up the example below}\\
\noindent \textit{Example. Course 3, Fall 2002, 36.} The number of claims in a period has a geometric distribution with mean $4$. The amount of each claim $X$ follows $\Pr(X=x) = 0.25, \ x=1,2,3,4.$ The number of claims and the claim amounts are independent. Let $S$ denote the aggregate claim amount in the period. Calculate $F_{S}(3)$.

\bigskip

\noindent \textit{Solution.} By definition, we have
\begin{align*}
F_{S}\left(3 \right) &= {\rm Pr}\left(\sum_{i=1}^N X_i \leq 3\right) = \sum_{n=0}^\infty {\rm Pr}\left(\sum_{i=1}^n X_i\leq 3|N=n\right){\rm Pr}(N=n) \\
&= \sum_n F^{\ast n} \left(3 \right) p_n = \sum_{n=0}^3 F^{\ast n}(3) p_n \\
&= p_0 + F^{\ast 1}(3) \ p_1 + F^{\ast 2}(3) \ p_2 + F^{\ast 3}(3) \ p_3
\end{align*}
Because $N$ has a geometric distribution with mean 4, we know that
\begin{align*}
p_n &= \frac{1}{1+\beta}
\left(\frac{\beta}{1+ \beta} \right)^n = \frac{1}{5} \left(\frac{4}{5} \right)^n
\end{align*}
For the claim severity distribution, recursively, we have
\begin{align*}
F^{\ast 1}(3) &= \Pr(X \le 3) = \frac{3}{4} \\
F^{\ast 2}(3) &= \sum_{y \le 3} F^{\ast 1} (3-y) f(y) = F^{\ast 1}(2)f(1) + F^{\ast 1}(1)f(2) \\
&= \frac{1}{4}\left[F^{\ast 1} (2) + F^{\ast 1}(1)\right] = \frac{1}{4}\left[{\rm Pr}(X\leq 2) + {\rm Pr}(X \leq 1) \right] \\
&= \frac{1}{4} \left(\frac{2}{4} + \frac{1}{4} \right) = \frac{3}{16}\\
F^{\ast 3}(3) &= \Pr(X_1+X_2 + X_3 \le 3) = \Pr(X_1=X_2=X_3=1) = \left(\frac{1}{4} \right)^3
\end{align*}
Notice that we did not need to recursively calculate $F^{\ast 3}(3)$ by recognizing that each $X \in \{1,2,3,4\}$, so the only way of obtaining $X_1+X_2+X_3 \leq 3$ is to have $X_1=X_2=X_3=1$. Additionally, for $n \geq 4$, $F^{\ast n} (3)=0$ since it is impossible for the sum of 4 or more $X$'s to be less than 3. For $n=0$, $F^{\ast 0}(3) = 1$ since the sum of 0 $X$'s is 0, which is always less than 3. Laying out the probabilities systematically,

\begin {center}
\begin{tabular}{c c c c}\hline
	$x$ & $F^{\ast 1}(x)$ & $F^{\ast 2}(x)$ & $F^{\ast 3}(x)$\\ \hline
	0 & & & \\
	1 & $\frac{1}{4}$ & $0$ & \\
	2 & $\frac{2}{4}$ & $\left( \frac{1}{4} \right)^2$ & \\
	3 & $\frac{3}{4}$ & $\frac{3}{16}$ & $\left( \frac{1}{4} \right)^3$ \\ \hline
\end{tabular}
\end{center}

Finally,
\begin{align*}
F_{S}(3) &= p_0 + F^{\ast 1}(3) \ p_1 + F^{\ast 2}(3) \ p_2 + F^{\ast 3}(3) \ p_3 \\
&= \frac{1}{5} + \frac{3}{4}\left(\frac{4}{25} \right) + \frac{3}{16} \left( \frac{16}{125} \right) + \frac{1}{64} \left( \frac{64}{625}\right) = 0.3456\\
\end{align*}
\begin{flushright}$\Box$\end{flushright}

\bigskip

%Page 12 of notes
%Here is another way to think about this problem. Lay them out systematically, as follows:
%\begin {center}
%\begin{tabular}{c c c c}\hline
%    $x$ & $F^{\ast 1}(x)$ & $F^{\ast 2}(x)$ & $F^{\ast 3}(x)$\\ \hline
%    0 & & & \\
%    1 & $\frac{1}{4}$ & $0$ & \\
%    2 & $\frac{2}{4}$ & $\left( \frac{1}{4} \right)^2$ & \\
%    3 & $\frac{3}{4}$ & $\frac{3}{16}$ & $\left( \frac{1}{4} \right)^3$ \\ \hline
%  \end{tabular}
%\end{center}
%\begin{flushright}$\Box$\end{flushright}


When $\mathrm{E}(N)$, one may also use the central limit theorem to approximate the distribution of $S$ as in the individual risk model. That is, $\frac{S - \mathrm{E}(S)}{\sqrt{\mathrm{Var}(S)}}$ approximately follows $N(0,1)$.

\bigskip

%{\color{red} Lisa clean up the example below}\\

%Page 6 of notes
\noindent \textit{Example. Exam 3, May 2000, 16.} You are given:

\begin{center}
  \begin{tabular}{ c | c  c }
    \hline
      & Mean & Standard Deviation\\ \hline
    Number of Claims & 8 & 3\\
    Individual Losses & 10,000 & 3,937\\
    \hline
  \end{tabular}
\end{center}
Using the normal approximation, determine the probability that the aggregate loss will exceed 150\% of the expected loss:

\bigskip

\begin{tabular}{ c c c c c}
(A) $\Phi (1.25)$ & (B) $\Phi (1.5)$ & (C) $1-\Phi (1.5)$ & (D) $1-\Phi (1.25)$ & (E) $1.5 \times \Phi (1)$
\end{tabular}

\bigskip

\noindent \textit{Solution.} To use the normal approximation, we must first find the mean and variance of the aggregate loss $S$
\begin{eqnarray*}
\mathrm{E~}S &=& \mu \ \mathrm{E~}N = 10,000(8) = 80,000\\
\mathrm{Var~}S &=& \sigma^2 \ \mathrm{E~}N + \mu^2 \ \mathrm{Var~}N\\
&=& 3937^2(8) + 10000^2 (3^2) = 1,023,999,752\\
\sqrt{\mathrm{Var~}S} &=& 31,999.996 \approx 32,000
\end{eqnarray*}
Then under the normal approximation, aggregate loss $S$ is approximately normal with mean 80,000 and standard deviation 32,000. The probability that $S$ will exceed 150\% of the expected aggregate loss is therefore
\begin{align*}
\Pr(S>1.5 \mathrm{E~}S) &= \Pr \left( \frac{S - \mathrm{E~} S}{\sqrt{\mathrm{Var~}S}} > \frac{1.5 \mathrm{E~}S - \mathrm{E~} S}{\sqrt{\mathrm{Var~}S}} \right) \\
&= \Pr \left( N(0,1) > \frac{0.5 \mathrm{E~}S}{\sqrt{\mathrm{Var~}S} } \right) \\
&= \Pr \left( N(0,1) > \frac{0.5(80,000)}{32,000} \right) = \Pr( N(0,1) > 1.25) \\
&= 1-\Phi(1.25) = 0.1056
\end{align*}
\begin{flushright}$\Box$\end{flushright}


%Page 13 of notes
\noindent \textit{Example. Course 3, November 2000, 32.} For an individual over $65$:
\begin{enumerate}
	\item[(i)] The number of pharmacy claims is a Poisson random variable with mean $25$.
	\item[(ii)] The amount of each pharmacy claim is uniformly distributed between $5$ and $95$.
	\item[(iii)] The amounts of the claims and the number of claims are mutually independent.
\end{enumerate}
Determine the probability that aggregate claims for this individual will exceed $2000$ using the normal approximation:

\bigskip

\begin{tabular}{ c c c c c}
	(A) $1-\Phi (1.33)$ & (B) $1-\Phi (1.66)$ & (C) $1-\Phi (2.33)$ & (D) $1-\Phi (2.66)$ & (E) $1-\Phi (3.33)$
\end{tabular}

\bigskip

\noindent \textit{Solution.} We have claim frequency $N \sim Poisson (\lambda = 25)$ and claim severity $X \sim U \left(5, 95 \right)$. To use the normal approximation, we need to find the mean and variance of the aggregate claims $S$. Note

\begin{center}
\begin{tabular}{lll}
$\mathrm{E~} N = 25$ & & $\mathrm{Var~} N = 25$\\
$\mathrm{E~}X = \frac{5+95}{2} = 50 = \mu$ & & $\mathrm{Var~}X = \frac{(95-5)^2}{12} = 675 = \sigma^2$\\
\end{tabular}
\end{center}
Then for $S$,
\begin{eqnarray*}
\mathrm{E~}S &=& \mu \ \mathrm{E~} N = 50(25) = 1,250\\
\mathrm{Var~}S &=& \sigma^2 \ \mathrm{E~}N + \mu^2 \ \mathrm{Var~}N\\
&=& 675 (25) + 50^2 (25) = 79,375
\end{eqnarray*}

Using the normal approximation, $S$ is approximately normal with mean 1,250 and variance 79,375. The probability that $S$ exceeds 2,000 is
\begin{align*}
\Pr(S>2,000) &= \Pr \left(\frac{S - \mathrm{E~} S}{\sqrt{\mathrm{Var~} S}} > \frac{2,000- \mathrm{E~} S}{\sqrt{\mathrm{Var~} S}} \right) \\
&= \Pr\left( N(0,1) > \frac{2,000-1,250}{\sqrt{79,375}} \right) \\
&= \Pr (N(0,1) > 2.662) = 1-\Phi(2.662) = 0.003884
\end{align*}
\begin{flushright}$\Box$\end{flushright}



\subsection{Stop-loss Insurance}

Insurance on the aggregate loss $S$, subjected to a deductible $d$, is called \textit{net stop-loss insurance}. The quantity
\begin{eqnarray*}
\mathrm{E}[(S-d)_+]
\end{eqnarray*}
is known as the \textit{net stop-loss premium}.

To calculate the net stop-loss premium, we have
\begin{eqnarray*}
\mathrm{E}(S-d)_+  & =& \int_{d}^{\infty} \left(1-F_S(s) \right) ds\\
&=& \int_{d}^{\infty}(s-d) f_{S}(s) ds ~~~~{\rm continuous}\\
&=& \sum_{s>d}(s-d) f_{S}(s) ds  ~~~~~{\rm discrete}\\
&=& \mathrm{E}(S) - \mathrm{E}(S\wedge d)\\
\end{eqnarray*}



%{\color{red} Lisa clean up the example below}\\

\noindent \textit{Example. Exam M, Fall 2005, 19.} In a given week, the number of projects that require you to work overtime has a geometric distribution with $\beta=2$. For each project, the distribution of the number of overtime hours in the week is as follows:

\bigskip

\begin{center}
\begin{tabular}{c c c } \hline
    $x$ &  & $f(x)$\\ \hline
    5 &  & 0.2 \\
    10 & & 0.3 \\
    20 & & 0.5\\ \hline
  \end{tabular}
\end{center}
\bigskip
The number of projects and the number of overtime hours are independent. You will get paid for overtime hours in excess of 15 hours in the week. Calculate the expected number of overtime hours for which you will get paid in the week.
\bigskip

\noindent \textit{Solution.} The number of projects in a week requiring overtime work has distribution $N \sim Geometric(\beta=2)$, while the number of overtime hours worked per project has distribution $X$ as described above. The aggregate number of overtime hours in a week is $S$ and we are therefore looking for
$$\mathrm{E~}(S-15)_+ = \mathrm{E~}S - \mathrm{E~}(S \wedge 15).$$

To find $\mathrm{E~}S = \mathrm{E~}X \ \mathrm{E~}N$, we have
\begin{eqnarray*}
\mathrm{E~}X &=& 5(0.2) + 10(0.3)+ 20(0.5)= 14 \\
\mathrm{E~}N &=& 2 \\
\Rightarrow \ \mathrm{E~}S &=& \mathrm{E~}X \ \mathrm{E~}N = 14(2) = 28
\end{eqnarray*}

To find $\mathrm{E~} (S \wedge 15) = 0 \Pr (S=0) + 5 \Pr(S=5) + 10 \Pr(S=10) + 15 \Pr(S \geq 15)$, we have
\begin{eqnarray*}
\Pr(S=0) &=& \Pr(N=0) = \frac{1}{1+\beta} = \frac{1}{3} \\
\Pr(S=5) &=& \Pr(X=5, \ N=1) = 0.2 \left(\frac{2}{9} \right)= \frac{0.4}{9}\\
\Pr(S=10) &=& \Pr(X=10, \ N=1) + \Pr(X_1=X_2=5, N=2) \\
&=& 0.3 \left(\frac{2}{9} \right) + (0.2)(0.2) \left( \frac{4}{27} \right)= 0.0726 \\
\Pr(S \geq 15) &=& 1 - \left(\frac{1}{3} + \frac{0.4}{9} + 0.0726 \right) = 0.5496\\
\Rightarrow \mathrm{E~}(S \wedge 15) &=& 0 \Pr (S=0) + 5 \Pr(S=5) + 10 \Pr(S=10) + 15 \Pr(S \geq 15) \\
&=& 0 \left( \frac{1}{3} \right) + 5
\left( \frac{0.4}{9} \right) + 10 (0.0726) + 15 (0.5496) = 9.193\\
\end{eqnarray*}
Therefore,
\begin{eqnarray*}
\mathrm{E~}(S-15)_+ &=& \mathrm{E~}S - \mathrm{E~}(S \wedge 15) \\
&=& 28 - 9.193 = 18.807
\end{eqnarray*}
\begin{flushright}$\Box$\end{flushright}




\textit{Recursive Net Stop-Loss Premium Calculation}. For the discrete case, this can be computed recursively as
\begin{eqnarray*}
\mathrm{E~}\left[ S-(j+1)h \right] _{+}=\mathrm{E~}\left[ ( S-jh )_{+} \right] -h \left( 1-F_S(jh)
\right) .
\end{eqnarray*}
This assumes that the support of $S$ is equally spaced over units of $h$.

To establish this, we assume that $h=1$. Now, on the left-hand side, we have $\mathrm{E~}\left[ S-(j+1) \right] _{+}=\mathrm{E~}S - \mathrm{E~}S\wedge (j+1)$. We can write
\begin{eqnarray*}
\mathrm{E~}S\wedge (j+1) = \sum_{x=0}^{j}xf_S(x) + (j+1)\Pr(S \ge j+1).
\end{eqnarray*}
Similarly
\begin{eqnarray*}
\mathrm{E~}S\wedge j = \sum_{x=0}^{j}xf_S(x) + j\Pr(S\ge j+1).
\end{eqnarray*}
With these, expressions, we have
\begin{eqnarray*}
\mathrm{E~}\left[ S-(j+1) \right] _{+} &-& \mathrm{E~}\left[ ( S-j )_{+} \right]  \\
&=&\left\{\mathrm{E~}S - \mathrm{E~}S\wedge (j+1) \right\}
-\left\{\mathrm{E~}S - \mathrm{E~}S\wedge j \right\} \\
&=&\left\{ \sum_{x=0}^{j}xf_S(x) + j\Pr(S\ge j+1) \right\}
- \left\{ \sum_{x=0}^{j}xf_S(x) + (j+1)\Pr(S \ge j+1) \right\} \\
&=& -\Pr(S\ge j+1) = -\{1 - F_{S}(j)\},
\end{eqnarray*}
as required.
\begin{flushright}$\Box$\end{flushright}


%
%
%\textit{Example. KPW Example 9.5, with units of h=25 dollars.}
%
%\begin{eqnarray*}
%\mathrm{E~}S &=& 314.5\\
%\mathrm{E~}(S-25)_+ &=& \mathrm{E~}S - 25 \Pr (S\ge 25)\\
%&=& 314.5 -25 (.95) = 290.75\\
%\mathrm{E~}(S-50)_+ &=& \mathrm{E~}(S-25)_+ - 25 \Pr (S\ge 50)\\
%&=& 290.75 - 25 (.935) = 267.375\\
%\end{eqnarray*}



%{\color{red} Lisa clean up the example below}\\


\noindent \textit{Exercise. Exam M, Fall 2005, 19 - Continued}. Recall that the goal of this question was to calculate $\mathrm{E~}(S-15)_+$. Note that the support of $S$ is equally spaced over units of 5, so this question can also be done recursively, using steps of $h=5$:

\begin{itemize}
\item Step 1:
	\begin{align*}
	\mathrm{E~}(S-5)_+ &= \mathrm{E~}S - 5 [1-\Pr(S \leq 0) ]\\ %\Pr (S\geq 5) \\
	&= 28 - 5 \left(1 - \frac{1}{3}\right) = \frac{74}{3}=24.6667
	\end{align*}
\item Step 2:
	\begin{align*}
	\mathrm{E~}(S-10)_+ &= \mathrm{E~}(S-5)_+ - 5 [1-\Pr(S \leq 5)]\\ %\Pr (S\ge 10) \\
	&= \frac{74}{3} - 5\left( 1 - \frac{1}{3} - \frac{0.4}{9}\right) = 21.555
	\end{align*}
\item Step 3:
	\begin{align*}
	\mathrm{E~}(S-15)_+ &= \mathrm{E~}(S-10)_+ - 5 [1-\Pr(S \leq 10)] \\ %\Pr (S\ge 15) \\
	&= \mathrm{E~}(S-10)_+ - 5\Pr (S\ge 15) \\
	&= 21.555 - 5 (0.5496) = 18.887
	\end{align*}
\end{itemize}
\begin{flushright}$\Box$\end{flushright}


\subsection{Analytic Results}


There are few combinations of claim frequency and severity distributions that result in an easy-to-compute distribution for aggregate losses. This section gives some simple examples. Analysts view these examples as too simple to be used in practice.

\bigskip
\textit{Example \#1} One has a closed-form expression for the aggregate loss distribution by assuming a geometric frequency distribution and an exponential severity distribution.

Assume that claim count $N$ is geometric with parameter $\beta$ such that $\mathrm{E}(N)=\beta$, and that claim amount $X$ is exponential with parameter $\theta$ such that $\mathrm{E}(X)=\theta$. Recall that the pgf of $N$ and the mgf of $X$ are:
\begin{align*}
P_N (z) &=\frac{1}{1- \beta (z-1)},\\
M_{X}(t) &=\frac{1}{1-\theta t}.
\end{align*}
Thus, the mgf of aggregate loss $S$ is
\begin{eqnarray}
M_{S}(t) &=& P_N [M_{X}(t)] = \frac{1}{1 - \beta \left( \frac{1}{1-\theta t} + 1\right)} \nonumber\\
&=& 1+ \frac{1}{1+\beta} ([1-\theta(1+\beta)z]^{-1}-1) \label{equ:geo1}\\
&=& \frac{1}{1+\beta}(1) +\frac{\beta}{1+\beta}
\left( \frac{1}{1-\theta (1+\beta)t}\right)   \label{equ:geo2}
\end{eqnarray}

From (\ref{equ:geo1}), we note that $S$ is equivalent to the compound distribution of $S=X^{*}_1+\cdots+X^{*}_{N^{*}}$, where $N^{*}$ is a Bernoulli with mean $\beta/(1+\beta)$ and $x^{*}$ is an exponential with mean $\theta(1+\beta)$. To see this, we examine the mgf of $S$:
\begin{align*}
M_{S}(t) = P_N [M_{X}(t)] = P_{N^{*}} [M_{X^{*}}(t)],
\end{align*}
where
\begin{align*}
P_{N^*} (z) &=1+ {\color{blue}\frac{\beta}{1+ \beta}} (z-1),\\
M_{X^*} (t) &=\frac{1}{1- {\color{blue}\theta(1+\beta)} t}.
\end{align*}

From (\ref{equ:geo2}), we note that $S$ is also equivalent to a 2-point mixture of 0 and $X^{*}$. Specifically,
\begin{eqnarray*}
S &=&
\left\{
\begin{array}{cl}
0 & {\rm with~ probability ~Pr}(N^*=0) = 1/(1+\beta) \\
Y^{*} & {\rm with~ probability ~Pr}(N^*=1) = \beta/(1+\beta)
\end{array}
\right..
\end{eqnarray*}
The distribution function of $S$ is:
\begin{eqnarray*}
\Pr(S=0) &=& \frac{1}{1+\beta}\\
\Pr(S>s) &=& \Pr(X^*>s) =\frac{\beta}{1+\beta} \exp\left( -\frac{s}{
\theta (1+\beta)}\right)
\end{eqnarray*}
with pdf
\begin{eqnarray*}
f_{S}(s) = \frac{\beta}{\theta (1+\beta)^2}\exp\left( -\frac{s}{
\theta (1+\beta)}\right).
\end{eqnarray*}
\begin{flushright}$\Box$\end{flushright}


\bigskip

\textit{Example \#2} Consider a collective risk model with an exponential severity and an arbitrary frequency distribution. Recall that if If $X_i\sim Exponential(\theta)$, then the sum of i.i.d. exponential, $S_n=X_1+\cdots+X_n$, has a Gamma distribution, i.e. $S_n\sim Gamma(n,\theta)$. This has cdf:
\begin{eqnarray*}
F_{X}^{\ast n}(s) &=& \Pr (S_n \le s) = \int_{0}^{s} \frac{1}{\Gamma(n)\theta^n}s^{n-1}\exp\left(-\frac{s}{\theta}\right) ds\\
&=& 1-\sum_{j=0}^{n-1}\frac{1}{j!}\left( \frac{s}{\theta}\right)^j e^{-s/\theta } .
\end{eqnarray*}
The last equality is derived by integration by parts.

For the aggregate loss distribution, we can interchange order of summations to get
\begin{eqnarray*}
F_{S}\left(s\right) &=& p_{0}+\sum_{n=1}^{\infty }p_n F_{X}^{\ast n}\left(s\right)\\
&=& 1 - \sum_{n=1}^{\infty }p_n \sum_{j=0}^{n-1}\frac{1}{j!}
\left( \frac{s}{\theta}\right)^j e^{-s/\theta }\\
&=& 1-e^{-s/\theta}\sum_{j=0}^{\infty} \frac{1}{j!}
\left( \frac{s}{\theta} \right)^j \overline{P}_j
\end{eqnarray*}
where $\overline{P}_j =p_{j+1}+p_{j+2}+\cdots = \Pr (N>j),$ the ``survival function'' of the claims count distribution.
\bigskip


\subsection{Tweedie Distribution}

In this section, we examine a particular compound distribution where the number of claims is a Poisson distribution and the amount of claims is a Gamma distribution. This specification leads to what is known as a Tweedie distribution. The Tweedie distribution has a mass probability at zero and a continuous component for positive values. Because of this feature, it is widely used in insurance claims modeling, where the zero mass is interpreted as no claims and the positive component as the amount of claims.

Specifically, consider the collective risk model $S=X_1+\cdots+X_N$. Suppose that $N$ has a Poisson distribution with mean $\lambda$, and each $X_i$ has a Gamma distribution shape parameter $\alpha$ and scale parameter $\gamma$. The Tweedie distribution is derived as the Poisson sum of gamma variables. To understand the distribution of $S$, we first examine the mass probability at zero. It is straightforward to see that the aggregate loss is zero when there is no claims occurred, thus:
$$f_S(0)={\rm Pr}(S=0)= {\rm Pr}(N=0)=e^{-\lambda}.$$
In addition, one notes that that $S$ conditional on $N_i=n$, denoted by $S_n=X_1+\cdots+X_n$, follows a gamma distribution with shape $n\alpha$ and scale $\gamma_i$. Thus, for $s>0$, the density of a Tweedie distribution can be calculated as
\begin{align*}
f_S(s)&=\sum_{n=1}^{+\infty} p_n f_{S_n}(s)\\
&=\sum_{n=1}^{\infty}e^{-\lambda_i}\frac{(\lambda_i)^n}{n!}\frac{1}{\gamma^{n\alpha}}y^{n\alpha-1}e^{-y\gamma}
\end{align*}
Thus, the Tweedie distribution can be thought of a mixture of zero and a positive valued distribution, which makes it a convenient tool for modeling insurance claims and for calculating pure premiums. The mean and variance of the Tweedie compound Poisson model are:
$${\rm E} (S)=\lambda\frac{\alpha}{\gamma}~~~~{\rm and}~~~~{\rm Var} (S)=\lambda\frac{\alpha(1+\alpha)}{\gamma^2}.$$

As another important feature, the Tweedie distribution is a special case of exponential dispersion models, a class of models used to describe the random component in generalized linear models. To see this, we consider the following reparameterizations:
\begin{equation*}
\lambda=\frac{\mu^{2-p}}{\phi(2-p)},~~~~\alpha=\frac{2-p}{p-1},~~~~\gamma=\phi(p-1)\mu^{p-1}
\end{equation*}
With the above relationships, one can show that the distribution of $S$ is
$$f_S(s)=\exp\left[\frac{1}{\phi}\left(\frac{-s}{(p-1)\mu^{p-1}}-\frac{\mu^{2-p}}{2-p}\right)+C(s;\phi)\right]$$
where
\begin{equation*}
C(s;\phi/\omega_i)=\left\{\begin{array}{ll}
                    \displaystyle 0 & {\rm if}~ y=0 \\
                   \displaystyle \ln \sum\limits_{n\ge 1} \left\{\frac{(1/\phi)^{1/(p-1)}y^{(2-p)/(p-1)}}{(2-p)(p-1)^{(2-p)/(p-1)}}\right\}^{n}\frac{1}{n!\Gamma(n(2-p)/(p-1))s} & {\rm if}~ y>0
                  \end{array}\right.
\end{equation*}
Hence, the distribution of $S$ belongs to the exponential family with parameters $\mu$, $\phi$, and $p\in(1,2)$, and we have
$${\rm E} (S)=\mu~~~~{\rm and}~~~~{\rm Var} (S)=\phi\mu^{p}$$
It is also worth mentioning the two limiting cases of the Tweedie model: $p\rightarrow 1$ results in the Poisson distribution and $p\rightarrow 2$ results in the gamma distribution. The Tweedie compound Poisson model accommodates the situations in between.




\section{Computing the Aggregate Claims Distribution}


Computing the distribution of aggregate losses is a difficult, yet important, problem. As we have seen, for both individual risk model and collective risk model, computing the distribution involves the evaluation of a $n$-fold convolution. To make the problem tractable, one strategy is to use a distribution that is easy to evaluate to approximate the aggregate loss distribution. For instance, normal distribution is a natural choice based on central limit theorem where parameters of the normal distribution can be estimated by matching the moments. This approach has its strength and limitations. The main advantage is the ease of computation. The disadvantage are: first, the size and direction of approximation error are unknown; second, the approximation may fail to capture some special features of the aggregate loss such as mass point at zero.

This section discusses two practical approaches to computing the distribution of aggregate loss, the recursive method and the simulation.

\subsection{Recursive Method}

The recursive method applies to compound models where the frequency component $N$ belongs to either $(a,b,0)$ or $(a,b,1)$ class and the severity component $X$ has a discrete distribution. For continuous $X$, a common practice is to first discretize the severity distribution and then the recursive method is ready to apply.

Assume that $N$ is in the $(a,b,1)$ class so that $p_{k}=\left( a+\frac{b}{k} \right) p_{k-1}, k = 2,3,\ldots$. Further assume that the support of $X$ is $\{0,1,\ldots,m\}$, discrete and finite. Then, the probability function of $S$ is:
\begin{align*}
f_{S}(s)&=\Pr (S=s) \\
&=\frac{1}{1-af_{X}(0)}\left\{ \left[ p_1 -(a+b)p_{0}\right]
f_X (s)+\sum_{x=1}^{s\wedge m}\left( a+\frac{bx}{s} \right) f_X (x)f_{S}(s-x)\right\}.
\end{align*}
If $N$ is in the $(a,b,0)$ class, then $p_1=(a+b)p_0$ and so
\[
f_S(s)=\frac{1}{1-af_X (0)}\left\{ \sum_{x=1}^{s\wedge m}\left( a+\frac{bx
}{s}\right) f_X (x)f_{S}(s-x)\right\}.
\]
\textit{Special Case}: If $N \sim$ Poisson with mean $\lambda$, then $a=0$ and $b=\lambda$, thus
\[
f_{S}(s)=\frac{\lambda }{s}\left\{ \sum_{x=1}^{s \wedge
m} x f_X (x) f_S (s-x)\right\} .
\]

%%Page 24 of notes
%
%\textit{Check}. We establish the result for the $(a,b,0)$ for $N$. To this end, recall $M_S (t) = P_N\left( M_X (t)\right)$, and
%\[
%P_S (z) = M_S \left(\ln z\right) = M_S (t) = P_N \left( M_X (\ln z)\right)
%= P_N \left( P_X (z)\right) = \sum_n p_n \left( P_X (z)\right)^n
%\]
%Taking derivatives, we have
%\[
%P_S '(z) = \sum_{n=1}^{\infty}n p_n P_X (z)^{n-1} P_X '(z)
%\]
%For $(a,b,0)$, we have the recursion
%\[
%n p_n = (na+b) p_{n-1} = a(n-1) p_{n-1} + (a+b) p_{n-1}
%\]
%Thus,
%\begin{eqnarray*}
%P_S '(z) &=& a \sum (n-1) p_{n-1} P_X (z)^{n-1} P_X '(z) + (a+b)\sum p_{n-1}
%P_X (z)^{n-1} P_X '(z)\\
%&=& a P_S '(z) P_X (z) + (a+b) P_S (z) P_X '(z)
%\end{eqnarray*}
%This yields
%\[
%P_S '(z)\left(1-a P_X (z)\right)=(a+b) P_S (z) P_X '(z)
%\]
%Evaluating this at $z=0$ yields
%\[
%f_S (1) \left( 1-a f_X (0)\right) =  (a+b) f_S(0) f_X (1)
%\]
%This establishes the result for $x=1$.
%
%Expand each and match powers  of $z$ to complete the proof. See KPW page 128 for details.
%\begin{flushright}$\Box$\end{flushright}
%
%%Page 25 of notes
%\[
%f_S (x) = \frac{1}{1-a f_X (0)}\{\sum_{y=1}^{x\wedge m}\left( a+\frac{by}{x}
%\right) f_X (y) f_S  (x-y) \}
%\]


%{\color{red} Lisa clean up the example below}\\


\noindent \textit{Example. SOA Fall 2002, 36.} The number of claims in a period $N$ has a geometric distribution with mean 4. The amount of each claim $X$ follows ${\rm Pr} (X = x) = 0.25$, for $x = 1,2,3,4$. The number of claims and the claim amount are independent. $S$ is the aggregate claim amount in the period. Calculate $F_S(3)$.

\bigskip

\noindent \textit{Solution.} The severity distribution $X$ follows
$$f_X (x) = \frac{1}{4}, \ \ x=1, 2, 3, 4.$$
The frequency distribution $N$ is geometric with mean 4, which is a member of the $(a,b,0)$ class with $b=0$, $a=\frac{\beta}{1+\beta} = \frac{4}{5}$, and $p_0 = \frac{1}{1+\beta} = \frac{1}{5}$. Thus, we can use the recursive method
\begin{eqnarray*}
f_S (x) &=& 1 \sum_{y=1}^{x\wedge m} (a+0) f_X (y) f_S (x-y) \\
&=& \frac{4}{5} \sum_{y=1}^{x\wedge m} f_X (y) f_S (x-y)
\end{eqnarray*}
Specifically, we have
\begin{eqnarray*}
f_S (0) &=& \Pr(N=0) = p_0=\frac{1}{5}\\
f_S (1) &=& \frac{4}{5}\sum_{y=1}^{1} f_X (y) f_S (1-y) = \frac{4}{5} f_X(1) f_S(0)\\
&=& \frac{4}{5}\left( \frac{1}{4}\right)\left(\frac{1}{5} \right) = \frac{1}{25}\\
f_S (2) &=&  \frac{4}{5}\sum_{y=1}^{2} f_X (y) f_S (2-y) = \frac{4}{5} \left[ f_X(1)f_S(1) + f_X(2) f_S(0) \right] \\
&=& \frac{4}{5}\left[ \frac{1}{4} \left( \frac{1}{25} + \frac{1}{5}\right) \right] =
\frac{4}{5}\left( \frac{6}{100}\right) = \frac{6}{125}\\
f_S (3) &=& \frac{4}{5} \left[ f_X(1) f_S(2) + f_X(2)f_S(1) + f_X(3) f_S(0) \right]\\
&=& \frac{4}{5}\left[ \frac{1}{4} \left( \frac{1}{25} + \frac{1}{5} +
\frac{6}{125}\right) \right] = \frac{1}{5}\left( \frac{5+25+6}{125}\right) = 0.0576\\
\Rightarrow \ F_S (3) &=& f_S (0)+f_S (1)+f_S (2)+f_S (2)+f_S (3) = 0.3456
\end{eqnarray*}
\begin{flushright}$\Box$\end{flushright}



\subsection{Simulation}

The distribution of aggregate loss can be evaluated using Monte Carlo simulation. The idea is one can calculate the empirical distribution of $S$ using a random sample. Blow we summarize the simulation procedures for the aggregate loss models.

\bigskip
\noindent\textcircled{1} Individual Risk Model $S_n=X_1+\cdots+X_n$
 \begin{itemize}
   \item For each $X_i$, $i=1,\ldots,n$, generate random sample of size $m$, denoted by $x_{ij}~(j=1,\ldots,m)$;
   \item Calculate the aggregate loss $s_j=x_{1j}+\ldots+x_{nj}$ for $j=1,\ldots,m$;
   \item We obtain a random sample of $S$, i.e. $\{s_1,\ldots,s_m\}$.
 \end{itemize}

\noindent\textcircled{2} Collective Risk Model $S=Y_1+\cdots+Y_N$
  \begin{itemize}
   \item Generate the number of claims $n_j$ from frequency distribution $N$;
   \item Given $n_j$, generate the amount of claims for each claim independently from $Y$, denoted by $y_{1},\ldots,y_{n_j}$;
   \item Calculate the aggregate loss $s_j=y_{1j}+\ldots+y_{n_j}$;
   \item Repeat the above three steps for $j=1,\ldots,m$;
   \item We obtain a random sample of $S$, i.e. $\{s_1,\ldots,s_m\}$.
 \end{itemize}

Given the random sample of $S$, the empirical distribution can be calculated as
$$\hat{F}_S(s)=\frac{1}{m}\sum_{i=1}^{m}I(s_i\leq s),$$
where $I(\cdot)$ is an indicator function. The empirical distribution $\hat{F}_S(s)$ will converge to ${F}_S(s)$ almost surely as $m\rightarrow \infty$.

The above procedure assumes that the parameters of the frequency and severity distributions are known. In practice, one would need to estimate these parameters from the data. For instance, the assumptions in the collective risk model suggest a two-stage estimation where a model is developed for the number of claims $N$ from the data on claim counts and a model is developed for the severity of claims $X$ from the data on the amount of claims.

\section{Effects of Coverage Modifications}

\subsection{Impact of Exposure on Frequency}

This section focuses on an individual risk model for claim counts. Consider the number of claims from a group of $n$ policies:
$$S=X_1+\cdots+X_n$$
where we assume $X_i$ are i.i.d. representing the number of claims from policy $i$. In this case, the exposure for the portfolio is $b$ using policy as exposure base. The pgf of $S$ is
\begin{align*}
P_S(z)&={\rm E}(z^S)={\rm E}\left(z^{\sum_{i=1}^nX_i}\right)\\
&=\prod_{i=1}^n{\rm E}(z^{X_i})=[P_X(z)]^n
\end{align*}

\textit{Special Case} Poisson. If $X_i\sim Poisson(\lambda)$, its pgf is $P_X(z)=e^{\lambda(z-1)}$. Then the pgf of $S$ is
  $$P_S(z)=[e^{\lambda(z-1)}]^n=e^{n\lambda(z-1)}.$$
So $S\sim Poisson(n\lambda)$.

\bigskip
\textit{Special Case} Negative binomial. If $X_i\sim NegBin(\beta,r)$, its pgf is $P_X(z)=[1-\beta(z-1)]^{-r}$. Then the pgf of $S$ is
  $$P_S(z)=[[1-\beta(z-1)]^{-r}]^n=[1-\beta(z-1)]^{-nr}.$$
So $S\sim NB(\beta,nr)$.


%{\color{red} Lisa clean up and finish the example on MLE below}\\

\bigskip

\noindent \textit{Example}. Assume that the number of claims for each vehicle is Poisson with mean $\lambda$. Given the following data on the observed number of claims for each household, calculate the MLE of $\lambda$.

\begin{center}
\begin{tabular}{|c|r|r|}
  \hline
  % after \\: \hline or \cline{col1-col2} \cline{col3-col4} ...
  Household ID & Number of vehicles & Number of claims \\ % # claims per household?!
  \hline
  1 & 2 & 0 \\
  2 & 1 & 2 \\
  3 & 3 & 2 \\
  4 & 1 & 0 \\
  5 & 1 & 1 \\
  \hline
\end{tabular}
\end{center}

\bigskip

\noindent \textit{Solution.} Each of the 5 households has number of exposures $b_j$ (number of vehicles) and number of claims $S_j$, $j=1,...,5$. Note for each household, the number of claims $S_j \sim Poisson (b_j \lambda)$. The likelihood function is
%For the 8 vehicles in total, each with number of claims $N_i \sim Poisson(\lambda)$, $i=1,...,8$, the likelihood function is
\begin{align*}
L(\lambda) &= \prod_{j=1}^5 \Pr(S_j=s_j) = \prod_{j=1}^5 \frac{e^{-b_j\lambda} (b_j \lambda)^{s_j}}{s_j!} \\
&= \left(\frac{e^{-2\lambda} (2 \lambda)^{0}}{0!} \right)
\left(\frac{e^{-1\lambda} (1 \lambda)^{2}}{2!} \right)
\left(\frac{e^{-3\lambda} (3 \lambda)^{2}}{2!} \right)
\left(\frac{e^{-1\lambda} (1 \lambda)^{0}}{0!} \right)
\left(\frac{e^{-1\lambda} (1 \lambda)^{1}}{1!} \right) \\
&\propto e^{-8\lambda} \lambda^5
\end{align*}
Taking the log-likehood, we have
\begin{align*}
l(\lambda) = \log L(\lambda) = -8\lambda + 5\log(\lambda)
\end{align*}
Setting the first derivative of the log-likehood to 0, we have
\begin{align*}
&l'(\lambda) = -8 + \frac{5}{\lambda} = 0 \\
\Rightarrow \ & 8 = \frac{5}{\hat{\lambda}} \ \Rightarrow \ \hat{\lambda} = \frac{5}{8}
\end{align*}

\begin{flushright}$\Box$\end{flushright}

If the exposure of the portfolio change from $n_1$ to $n_2$, we can establish the following relation between the aggregate claim counts:
$$P_{S_2}(z)=[P_X(z)]^{n_2}=[P_X(z)^{n_1}]^{n_2/n_1}=P_{S_1}(z)^{n_2/n_1}.$$

\subsection{Impact of Deductibles on Claim Frequency}

This section examine the effect of deductible on claim frequency. Intuitively, there will be fewer claims filed when a policy deductible is imposed because a loss below deductible might not result in a claim. Even if an insured does file a claim, this may not result in a payment by the policy, since the claim may be denied or the loss amount may ultimately be determined to be below deductible. Let $N^L$ denote the number of loss (i.e. the number of claims with no deductible), and $N^P$ denote the number of payments when a deductible $d$ is imposed. Our goal is to identify the distribution of $N^P$ given the distribution of $N^L$.  We show below that the relationship between $N^L$ and $N^P$ can be established within an aggregate risk model framework.

Note that sometimes changes in deductible will affect policyholder behavior. We assume that this is not the case, i.e. the distribution of losses for both frequency and severity remain unchanged when the deductible changes.

Given there are $N^L$ losses, let $X_1,X_2\ldots,X_{N^L}$ be the associated amount of losses. For $j=1,\ldots,N^L$, define
\begin{eqnarray*}
I_j&=&
\left \{
\begin{array}{cc}
1 & \text{if} ~X_j>d\\
0 & \text{otherwise}\\
\end{array}
\right..
\end{eqnarray*}
Then we establish
$$N^P=I_1+I_2+\cdots+I_{N_L}.$$


Note that conditioning on $N^L$, the distribution of $N^P \sim Binomial (N^L, v)$, where $v=\Pr(X>d)$. Thus, given $N^L$,
\begin{eqnarray*}
\mathrm{E}\left(z^{N^P}|N^L\right)&=&\left[ 1+v(z-1)\right]^{N^L}
\end{eqnarray*}
So the p.g.f. of $N^P$ is
\begin{eqnarray*}
P_{N^P}(z)&=&\mathrm{E}_{N^P}\left(z^{N^P}\right)=\mathrm{E}_{N^L}\left[\mathrm{E}_{N^P}\left(z^{N^P}|N^L\right)\right]\\
&=&\mathrm{E}_{N^L}\left[(1+v(z-1))^{N^L}\right]\\
&=&P_{N^L}\left(1+v(z-1)\right)
\end{eqnarray*}
Thus, we can write the pgf of $N^P$ as the pgf of $N^L$, evaluated at a new argument $z^* =  1+v(z-1)$, that is, $P_{N^P}(z)=P_{N^L}(z^*)$.




\bigskip
\textit{Special Cases:}
\begin{enumerate}
\item $N^L\sim Poisson (\lambda)$. The pgf of $N^L$ is $P_{N^L}=\exp(\lambda(z-1))$. Thus the pgf of $N^P$ is
\begin{eqnarray*}
P_{N^P}(z)&=&\exp\left( \lambda(1+v(z-1)-1)\right)\\
&=&\exp(\lambda v(z-1))\sim Poisson (\lambda v)
\end{eqnarray*}
So the payment number has the same distribution as the loss number but with the expected number of payments equal to $\lambda v = \lambda \Pr(X>d)$.


\item $N^L \sim NegBin(\beta, r)$. The pgf of $N^L$ is $P_{N^{L}}\left( z\right) =\left[ 1-\beta \left( z-1\right)\right]^{-r}$.
\begin{eqnarray*}
P_{N^P}(z)&=&\left( 1-\beta (1+v(z-1)-1)\right)^{-r}\\
&=&\left( 1-\beta v(z-1)\right)^{-r} \:\:\:\sim NegBin(\beta v, r)
\end{eqnarray*}
\end{enumerate}
So the payment number has the same distribution as the loss number but with parameters $\beta v$ and $r$.


%{\color{red} Lisa clean up and finish the example below}\\


\bigskip

\noindent \textit{Example.} Suppose that loss amounts $X_i\sim Pareto(\alpha=4,\ \theta=150)$. You are given that the loss frequency is $N^L\sim Poisson(\lambda)$ and the payment frequency distribution $N^{P_1}\sim Poisson (0.4)$ with $d_1=30$. Find the distribution of $N^{P_2}$ with $d_2=100$.

\bigskip

\noindent \textit{Solution.} Because the loss frequency $N^L$ is Poisson, we can relate the means of the loss distribution $N^L$ and the first payment distribution $N^{P_1}$ as $0.4 = v_1 \lambda$, where
\begin{align*}
&v_1 = \Pr(X > 30) = \left( \frac{150}{30+150}\right)^4=\left( \frac{5}{6}\right)^4 \\
\Rightarrow \ & \lambda = 0.4 \left( \frac{6}{5} \right)^4
\end{align*}
With this, we can assess the second payment distribution $N^{P_2}$ as being Poisson with mean $\lambda_2 = \lambda v_2$, where
\begin{align*}
& v_2 = \Pr(X>100)=\left( \frac{150}{100+150}\right)^4=\left( \frac{3}{5}\right)^4 \\
\Rightarrow \ & \lambda_2 = \lambda v_2 = 0.4\left( \frac{6}{5} \right)^4 \left( \frac{3}{5} \right)^4 = 0.1075
\end{align*}
\begin{flushright}$\Box$\end{flushright}

%Repeat with the negative binomial distribution.
\noindent \textit{Follow-Up.} Now suppose instead that the loss frequency is $N^L \sim NegBin(\beta,\ r)$ and for deductible $d_1=30$, the payment frequency $N^{P_1}$ is negative binomial with mean $0.4$. Find the mean of the payment frequency $N^{P_2}$ with deductible $d_2=100$.

\bigskip

\noindent \textit{Solution.} Because the loss frequency $N^L$ is negative binomial, we can relate the parameter $\beta$ of the $N^L$ distribution and the parameter $\beta_1$ of the first payment distribution $N^{P_1}$ using $\beta_1 = \beta v_1$, where $$v_1 = \Pr(X > 30) = \left( \frac{5}{6} \right)^4$$ Thus, the mean of $N^{P_1}$ and the mean of $N^L$ are related
\begin{align*}
&0.4 =  r \beta_1 = r \left(\beta v_1\right) \\
\Rightarrow \ & r\beta = \frac{0.4}{v_1} = 0.4 \left(\frac{6}{5} \right)^4
\end{align*}
Note that $v_2 = \Pr(X > 100) = \left( \frac{3}{5}\right)^4$ as in the original question. Then the second payment frequency distribution is $N^{P_2} \sim NegBin(\beta v_2, \ r)$ with mean
\begin{align*}
r (\beta v_2) = (r \beta) v_2 = 0.4 \left( \frac{6}{5}\right)^4 \left( \frac{3}{5} \right)^4 = 0.1075
\end{align*}
\begin{flushright}$\Box$\end{flushright}
\bigskip

Next we examine the more general case where $N^L$ is a zero-modified distribution. Recall that a modified distribution is defined in terms of an unmodified one. That is,
\begin{align*}
p_k^M = c~p_k^0, {~\rm for~} k=1,2,3,\ldots,  {~\rm with~}c = \frac{1-p_0^M}{1-p_0^0}.
\end{align*}
In the case that $p_0^M=0$, we call this a ``truncated'' distribution at zero, or $ZT$. For other arbitrary values of $p_0^M$, this is a zero-modified, or $ZM$, distribution. The pgf for the modified distribution is shown as
\begin{eqnarray*}
P^M(z) = 1-c+c~P^0(z).
\end{eqnarray*}


%Check:
%\begin{eqnarray*}
%P^M(z) &=& \sum_{k=0}^{\infty} p_k^M~z^k = p_0^M + c (\sum_{k=0}^{\infty} p_k^0~z^k - p_0^0)\\
%&=&p_0^M + c(P^0(z)-p^0_0)\\
%&=& 1 - c + cP^0(z),
%\end{eqnarray*}
%as required. By taking $z=0$, we have on the left-hand side, that $P^M(0)=p_0^M$. On the right-hand side, we have $1-c+cP^0(0)=1-c+cp^0_0$. Equating the two yields
%\begin{eqnarray*}
%c = \frac{1-p_0^M}{1-p^0_0}
%\end{eqnarray*}
%as noted above.
%\bigskip
%

When $N^L$ follows a zero-modified distribution, the distribution of $N^P$ is established using the same relation $P_{N^P}(z)=P_{N^L}\left(1+v(z-1)\right)$.


\bigskip
\textit{Special Cases:}
\begin{enumerate}
\item $N^{L}$ is a ZM-Poisson with parameters $\lambda$ and $p_0^{M}$. The pgf of $N^L$ is
$$P_{N^{L}}(z)=1-\cfrac{1-p_0^{M}}{1-\exp(-\lambda)}+\cfrac{1-p_0^{M}}{1-\exp(-\lambda)}\exp[\lambda(z-1)].$$
Thus the pgf of $N^P$ is
$$P_{N^{L}}(z)=1-\cfrac{1-p_0^{M}}{1-\exp(-\lambda)}+\cfrac{1-p_0^{M}}{1-\exp(-\lambda)}\exp[\lambda v(z-1)].$$
So the number of payments is also a ZM-Poisson distribution with parameters $\lambda v$ and $p_0^{M}$. The probability at zero can be evaluated using
${\rm Pr}(N^P=0) = P_{N^P}(0)$.


\item $N^{L}$ is a ZM-NegBin with parameters $\beta$, $r$, and $p_0^{M}$. The pgf of $N^L$ is
$$P_{N^{L}}(z)=1-\cfrac{1-p_0^{M}}{1-(1+\beta)^{-r}}+\cfrac{1-p_0^{M}}{1-(1+\beta)^{-r}}\left[ 1-\beta \left( z-1\right)\right]^{-r}.$$
Thus the pgf of $N^P$ is
$$P_{N^{L}}(z)=1-\cfrac{1-p_0^{M}}{1-(1+\beta)^{-r}}+\cfrac{1-p_0^{M}}{1-(1+\beta)^{-r}}\left[ 1-\beta v\left( z-1\right)\right]^{-r}.$$
So the number of payments is also a ZM-NegBin distribution with parameters $\beta v$, $r$, and $p_0^{M}$. Similarly, the probability at zero can be evaluated using
${\rm Pr}(N^P=0) = P_{N^P}(0)$.
\end{enumerate}


%{\color{red} Lisa given an numerical example below}\\

\noindent \textit{Example.} Aggregate losses are modeled as follows:
\begin{enumerate}
\item[(i)] The number of losses follows a zero-modified Poisson distribution with $\lambda=3$ and $p_0^M = 0.5$.
\item[(ii)] The amount of each loss has a Burr distribution with $\alpha=3, \theta=50, \gamma=1$.
\item[(iii)] There is a deductible of $d=30$ on each loss.
\item[(iv)] The number of losses and the amounts of the losses are mutually independent.
\end{enumerate}
Calculate $\mathrm{E~} N^P$ and $\mathrm{Var~} N^P$.

\bigskip

\noindent \textit{Solution.} Since $N^L$ follows a ZM-Poisson distribution with parameters $\lambda$ and $p_0^M$, we know that $N^P$ also follows a ZM-Poisson distribution, but with parameters $\lambda v$ and $p_0^M$, where

$$v = \Pr(X>30) = \left( \frac{1}{1+(30/50)} \right)^3 = 0.2441$$

Thus, $N^P$ follows a ZM-Poisson distribution with parameters $\lambda^\ast = \lambda v= 0.7324$ and $p_0^M = 0.5$. Finally,
\begin{align*}
\mathrm{E~} N^P &= (1-p_0^M) \frac{\lambda^\ast}{1-e^{-\lambda^\ast}} = 0.5 \left( \frac{0.7324}{1-e^{-0.7324}} \right) \\
&= 0.7053 \\
\mathrm{Var~} N^P &= (1-p_0^M) \left( \frac{\lambda^\ast[1-(\lambda^\ast + 1) e^{-\lambda^\ast}]}{(1-e^{-\lambda^\ast})^2} \right) + p_0^M(1-p_0^M) \left(\frac{\lambda^\ast}{1-e^{-\lambda^\ast}} \right)^2 \\
&= 0.5 \left( \frac{0.7324(1-1.7324 e^{-0.7324})}{(1-e^{-0.7324})^2} \right) + 0.5^2 \left( \frac{0.7324}{1-e^{-0.7324}} \right)^2 \\
&= 0.7244
\end{align*}
\begin{flushright}$\Box$\end{flushright}


\subsection{Impact of Policy Modifications on Aggregate Claims}


In this section, we examine how the change in deductibles affect aggregate payments from an insurance portfolio. We assume that policy limits, coinsurance, and inflation have no effect on the frequency of payments made by an insurer. As in the previous section, we further assume that deductible changes do not impact the distribution of losses for both frequency and severity.

Recall the notation $N^L$ for the number of losses. With ground-up loss $X$ and policy deductible $d$, we use
$N^P = I(X_1>d) + \cdots + I(X_{N^L}>d)$ for the number of payments.
Also, define the amount of payment on a per-loss basis as
\begin{eqnarray*}
    X^{L}&=\left\{
      \begin{array}{cc}
        0 & X<\cfrac{d}{1+r} \\
        \alpha[(1+r)X-d] & \cfrac{d}{1+r}\leq X<\cfrac{u}{1+r} \\
        \alpha(u-d) &  X \ge \cfrac{u}{1+r}\\
      \end{array}
\right.,
\end{eqnarray*}
and the the amount of payment on a per-payment basis as
\begin{eqnarray*}
    X^{P}&=\left\{
      \begin{array}{cc}
        {\rm Undefined} & X<\cfrac{d}{1+r} \\
        \alpha[(1+r)X-d] & \cfrac{d}{1+r}\leq X<\cfrac{u}{1+r} \\
        \alpha(u-d) &  X \ge \cfrac{u}{1+r}\\
      \end{array}
\right..
\end{eqnarray*}
In the above, $r$, $u$, $\alpha$ represents the inflation rate, policy limit, and coinsurance, respectively. Hence, aggregate costs (payment amounts) can be expressed either on a per loss or per
payment basis:
\begin{eqnarray*}
S &=& X^L_1 + \cdots + X^L_{N^L} \\
&=&X^P_1 + \cdots + X^P_{N^P} .
\end{eqnarray*}
The fundamentals regarding collective risk models are ready to apply. For instance, we have:
\begin{align*}
  {\rm E}(S) &= {\rm E}\left(N^L\right) {\rm E}\left(X^L\right) = {\rm E}\left(N^P\right) {\rm E}\left(X^P\right)\\
  {\rm Var}(S) &= {\rm E}\left(N^L\right) {\rm Var}\left(X^L\right) + \left[{\rm E}\left(X^L\right)\right]^2 {\rm Var}(N^L) \\
  &= {\rm E}\left(N^P\right) {\rm Var}\left(X^P\right) + \left[{\rm E}\left(X^P\right)\right]^2 {\rm Var}(N^P)\\
  M_S(z)&=P_{N^L}\left[M_{X^L}(z)\right]=P_{N^P}\left[M_{X^P}(z)\right]
\end{align*}


%{\color{red} Lisa clean example below}\\

\bigskip

%Page 31 of notes
\noindent \textit{Example. Course 3, November 2001, 6.} A group dental policy
has a negative binomial claim count distribution with mean 300 and variance 800. Ground-up severity is given by the following table:
\bigskip
\begin{center}
  \begin{tabular}{ c | c }
    \hline
      Severity & Probability\\ \hline
    40 & 0.25\\
    80 & 0.25\\
    120 & 0.25\\
    200 & 0.25\\
    \hline
  \end{tabular}
\end{center}
\bigskip
You expect severity to increase 50\% with no change in frequency. You decide to impose a per claim deductible of 100. Calculate the expected total claim payment after these changes.

\bigskip

\noindent \textit{Solution.} The cost per loss with a 50\% increase in severity and a 100 deductible per claim is
\begin{eqnarray*}
Y^L &=&
\left\{
\begin{array}{cc}
0 & 1.5x<100 \\
1.5x-100 & 1.5x\ge 100\\
\end{array}
\right.
\end{eqnarray*}
This has expectation
\begin{align*}
\mathrm{E~}Y^L &= \frac{1}{4} \left[ \left(1.5(40)-100\right)_+ +
\left(1.5(80)-100\right)_+ + \left(1.5(120)-100\right)_+ +
\left(1.5(200)-100\right)_+ \right]  \\
&= \frac{1}{4}\left[ (60-100)_+ + (120-100)_+ + (180-100)_+ + (300-100)_+\right] \\
&= \frac{1}{4}\left[ 0 + 20 + 80 + 200 \right] = 75
\end{align*}
Thus, the expected aggregate loss is
$$\mathrm{E~}S=(\mathrm{E~}N) \left( \mathrm{E~}Y^L \right)= 300 (75) = 22,500
.$$

%Page32 of notes
\noindent \textit{Follow-Up.} What is $\mathrm{Var~}S$?

\bigskip

\noindent On a per loss basis, we have
\begin{align*}
\mathrm{Var~}S &= \left(\mathrm{E~}N \right) \left( \mathrm{Var~} Y^L \right) + \left[ \mathrm{E~} Y^L \right]^2 \left(\mathrm{Var~} N \right)
\end{align*}
where $\mathrm{E~}N = 300$ and $\mathrm{Var~} N = 800$. We find
\begin{align*}
&\mathrm{E} \left[ (Y^L)^2 \right] = \frac{1}{4} \left[ 0^2 + 20^2 + 80^2 + 200^2 \right] = 11,700 \\
\Rightarrow \ & \mathrm{Var~} Y^L = \mathrm{E} \left[ (Y^L)^2 \right] - \left( \mathrm{E~}Y^L \right)^2 = 11,700 - 75^2 = 6,075
\end{align*}
Thus, the variance of the aggregate claim payment is
\begin{eqnarray*}
\mathrm{Var~}S &=& 300(6,075) + 75^2 (800) = 6,322,500
\end{eqnarray*}
\begin{flushright}$\Box$\end{flushright}
\textit{Alternative Method: Using the Per Payment Basis.} Previously, we calculated the expected total claim payment by multiplying the expected number of losses by the expected payment \emph{per loss}. Recall that we can also multiply the expected number of payments by the expected payment \emph{per payment}. In this case, we have
\[S=Y_1^P + \cdots + Y_{N_P}^P \]
The probability of a payment is
\[v=\Pr(1.5X \ge 100)=\Pr(X \ge 66.\bar{6})=\frac{3}{4} .\]
Thus, the number of payments, $N^P$ has a negative binomial distribution
with mean
$$\mathrm{E~}N^P=300 \left(\frac{3}{4} \right)=225$$
The cost per payment is
\begin{eqnarray*}
Y^P &=&
\left\{
\begin{array}{cc}
\text{undefined} & 1.5x<100 \\
1.5x-100 & 1.5x\ge 100\\
\end{array}
\right.
\end{eqnarray*}
This has expectation
$$\mathrm{E~}Y^P=\frac{\mathrm{E~}Y^L}{\Pr(1.5X > 100)}=
\frac{\mathrm{E~}Y^L}{v}=\frac{75}{(3/4)}=100$$
Thus, as before, the expected aggregate loss is
$$\mathrm{E~}S=\left(\mathrm{E~}Y^P\right) \left(\mathrm{E~}N^P\right) =
100(225)=22,500$$
\begin{flushright}$\Box$\end{flushright}




%{\color{red} Lisa give another example on per-payment variable, use SOA sample question Q109, follow above example}\\

\noindent \textit{Example. SOA Sample Question, 109.} A company insures a fleet of vehicles. Aggregate losses have a compound Poisson distribution. The expected number of losses is 20. Loss amounts, regardless of vehicle type, have exponential distribution with $\theta=200$. To reduce the cost of the insurance, two modifications are to be made:
\begin{enumerate}
	\item[(i)] A certain type of vehicle will not be insured. It is estimated that this will reduce loss frequency by 20\%.
	\item[(ii)] A deductible of 100 per loss will be imposed.
\end{enumerate}
Calculate the expected aggregate amount paid by the insurer after the modifications.

\bigskip

\noindent \textit{Solution.} On a per loss basis, we have a 100 deductible. Thus, the expectation per loss is
\begin{align*}
\mathrm{E~} Y^L &= E[(X-100)_+] = E(X) - E(X\wedge 100) \\
&= 200 - 200(1-e^{-100/200}) = 121.31
\end{align*}
Loss frequency has been reduced by 20\%, resulting in an expected number of losses
$$\mathrm{E~}N^L = 0.8(20) = 16$$
Thus, the expected aggregate amount paid after the modifications is
$$\mathrm{E~}S = \left(\mathrm{E~}Y^L \right) \left( \mathrm{E~} N^L\right) = 121.31(16) = 1,941$$
\begin{flushright}$\Box$\end{flushright}
\textit{Alternative Method: Using the Per Payment Basis.} We can also use the per payment basis to find the expected aggregate amount paid after the modifications. For the per payment severity,
\begin{align*}
\mathrm{E~} Y^P = \frac{\mathrm{E~} Y^L}{\Pr(X > 100)} = \frac{200 - 200(1-e^{-100/200})}{e^{-100/200}} = 200
\end{align*}
This is not surprising -- recall that the exponential distribution is memoryless, so the expected claim amounts paid in excess of 100 is still exponential with mean 200. Now we look at the payment frequency. With the deductible of 100, the probability that a payment occurs is $\Pr(X > 100) = e^{-100/200}$ Thus,
$$\mathrm{E~} N^P = 16 e^{-100/200} = 9.7$$
Putting this together, we produce the same answer using the per payment basis as the per loss basis from earlier
$$\mathrm{E~}S = \left( \mathrm{E~} Y^P \right) \left( \mathrm{E~} N^P \right) = 200(9.7) = 1,941$$
\begin{flushright}$\Box$\end{flushright}

\end{document}
