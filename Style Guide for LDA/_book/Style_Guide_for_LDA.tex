\documentclass[]{book}
\usepackage{lmodern}
\usepackage{amssymb,amsmath}
\usepackage{ifxetex,ifluatex}
\usepackage{fixltx2e} % provides \textsubscript
\ifnum 0\ifxetex 1\fi\ifluatex 1\fi=0 % if pdftex
  \usepackage[T1]{fontenc}
  \usepackage[utf8]{inputenc}
\else % if luatex or xelatex
  \ifxetex
    \usepackage{mathspec}
  \else
    \usepackage{fontspec}
  \fi
  \defaultfontfeatures{Ligatures=TeX,Scale=MatchLowercase}
\fi
% use upquote if available, for straight quotes in verbatim environments
\IfFileExists{upquote.sty}{\usepackage{upquote}}{}
% use microtype if available
\IfFileExists{microtype.sty}{%
\usepackage{microtype}
\UseMicrotypeSet[protrusion]{basicmath} % disable protrusion for tt fonts
}{}
\usepackage[margin=1in]{geometry}
\usepackage{hyperref}
\hypersetup{unicode=true,
            pdftitle={Style Guide for Loss Data Analytics},
            pdfauthor={An open text authored by the Actuarial Community},
            pdfborder={0 0 0},
            breaklinks=true}
\urlstyle{same}  % don't use monospace font for urls
\usepackage{natbib}
\bibliographystyle{apalike}
\usepackage{color}
\usepackage{fancyvrb}
\newcommand{\VerbBar}{|}
\newcommand{\VERB}{\Verb[commandchars=\\\{\}]}
\DefineVerbatimEnvironment{Highlighting}{Verbatim}{commandchars=\\\{\}}
% Add ',fontsize=\small' for more characters per line
\usepackage{framed}
\definecolor{shadecolor}{RGB}{248,248,248}
\newenvironment{Shaded}{\begin{snugshade}}{\end{snugshade}}
\newcommand{\KeywordTok}[1]{\textcolor[rgb]{0.13,0.29,0.53}{\textbf{#1}}}
\newcommand{\DataTypeTok}[1]{\textcolor[rgb]{0.13,0.29,0.53}{#1}}
\newcommand{\DecValTok}[1]{\textcolor[rgb]{0.00,0.00,0.81}{#1}}
\newcommand{\BaseNTok}[1]{\textcolor[rgb]{0.00,0.00,0.81}{#1}}
\newcommand{\FloatTok}[1]{\textcolor[rgb]{0.00,0.00,0.81}{#1}}
\newcommand{\ConstantTok}[1]{\textcolor[rgb]{0.00,0.00,0.00}{#1}}
\newcommand{\CharTok}[1]{\textcolor[rgb]{0.31,0.60,0.02}{#1}}
\newcommand{\SpecialCharTok}[1]{\textcolor[rgb]{0.00,0.00,0.00}{#1}}
\newcommand{\StringTok}[1]{\textcolor[rgb]{0.31,0.60,0.02}{#1}}
\newcommand{\VerbatimStringTok}[1]{\textcolor[rgb]{0.31,0.60,0.02}{#1}}
\newcommand{\SpecialStringTok}[1]{\textcolor[rgb]{0.31,0.60,0.02}{#1}}
\newcommand{\ImportTok}[1]{#1}
\newcommand{\CommentTok}[1]{\textcolor[rgb]{0.56,0.35,0.01}{\textit{#1}}}
\newcommand{\DocumentationTok}[1]{\textcolor[rgb]{0.56,0.35,0.01}{\textbf{\textit{#1}}}}
\newcommand{\AnnotationTok}[1]{\textcolor[rgb]{0.56,0.35,0.01}{\textbf{\textit{#1}}}}
\newcommand{\CommentVarTok}[1]{\textcolor[rgb]{0.56,0.35,0.01}{\textbf{\textit{#1}}}}
\newcommand{\OtherTok}[1]{\textcolor[rgb]{0.56,0.35,0.01}{#1}}
\newcommand{\FunctionTok}[1]{\textcolor[rgb]{0.00,0.00,0.00}{#1}}
\newcommand{\VariableTok}[1]{\textcolor[rgb]{0.00,0.00,0.00}{#1}}
\newcommand{\ControlFlowTok}[1]{\textcolor[rgb]{0.13,0.29,0.53}{\textbf{#1}}}
\newcommand{\OperatorTok}[1]{\textcolor[rgb]{0.81,0.36,0.00}{\textbf{#1}}}
\newcommand{\BuiltInTok}[1]{#1}
\newcommand{\ExtensionTok}[1]{#1}
\newcommand{\PreprocessorTok}[1]{\textcolor[rgb]{0.56,0.35,0.01}{\textit{#1}}}
\newcommand{\AttributeTok}[1]{\textcolor[rgb]{0.77,0.63,0.00}{#1}}
\newcommand{\RegionMarkerTok}[1]{#1}
\newcommand{\InformationTok}[1]{\textcolor[rgb]{0.56,0.35,0.01}{\textbf{\textit{#1}}}}
\newcommand{\WarningTok}[1]{\textcolor[rgb]{0.56,0.35,0.01}{\textbf{\textit{#1}}}}
\newcommand{\AlertTok}[1]{\textcolor[rgb]{0.94,0.16,0.16}{#1}}
\newcommand{\ErrorTok}[1]{\textcolor[rgb]{0.64,0.00,0.00}{\textbf{#1}}}
\newcommand{\NormalTok}[1]{#1}
\usepackage{longtable,booktabs}
\usepackage{graphicx,grffile}
\makeatletter
\def\maxwidth{\ifdim\Gin@nat@width>\linewidth\linewidth\else\Gin@nat@width\fi}
\def\maxheight{\ifdim\Gin@nat@height>\textheight\textheight\else\Gin@nat@height\fi}
\makeatother
% Scale images if necessary, so that they will not overflow the page
% margins by default, and it is still possible to overwrite the defaults
% using explicit options in \includegraphics[width, height, ...]{}
\setkeys{Gin}{width=\maxwidth,height=\maxheight,keepaspectratio}
\IfFileExists{parskip.sty}{%
\usepackage{parskip}
}{% else
\setlength{\parindent}{0pt}
\setlength{\parskip}{6pt plus 2pt minus 1pt}
}
\setlength{\emergencystretch}{3em}  % prevent overfull lines
\providecommand{\tightlist}{%
  \setlength{\itemsep}{0pt}\setlength{\parskip}{0pt}}
\setcounter{secnumdepth}{5}
% Redefines (sub)paragraphs to behave more like sections
\ifx\paragraph\undefined\else
\let\oldparagraph\paragraph
\renewcommand{\paragraph}[1]{\oldparagraph{#1}\mbox{}}
\fi
\ifx\subparagraph\undefined\else
\let\oldsubparagraph\subparagraph
\renewcommand{\subparagraph}[1]{\oldsubparagraph{#1}\mbox{}}
\fi

%%% Use protect on footnotes to avoid problems with footnotes in titles
\let\rmarkdownfootnote\footnote%
\def\footnote{\protect\rmarkdownfootnote}

%%% Change title format to be more compact
\usepackage{titling}

% Create subtitle command for use in maketitle
\newcommand{\subtitle}[1]{
  \posttitle{
    \begin{center}\large#1\end{center}
    }
}

\setlength{\droptitle}{-2em}
  \title{Style Guide for Loss Data Analytics}
  \pretitle{\vspace{\droptitle}\centering\huge}
  \posttitle{\par}
  \author{An open text authored by the Actuarial Community}
  \preauthor{\centering\large\emph}
  \postauthor{\par}
  \predate{\centering\large\emph}
  \postdate{\par}
  \date{2018-06-12}

\usepackage{booktabs}

\usepackage{amsthm}
\newtheorem{theorem}{Theorem}[chapter]
\newtheorem{lemma}{Lemma}[chapter]
\theoremstyle{definition}
\newtheorem{definition}{Definition}[chapter]
\newtheorem{corollary}{Corollary}[chapter]
\newtheorem{proposition}{Proposition}[chapter]
\theoremstyle{definition}
\newtheorem{example}{Example}[chapter]
\theoremstyle{definition}
\newtheorem{exercise}{Exercise}[chapter]
\theoremstyle{remark}
\newtheorem*{remark}{Remark}
\newtheorem*{solution}{Solution}
\begin{document}
\maketitle

{
\setcounter{tocdepth}{1}
\tableofcontents
}
\chapter{Chapter Structure}\label{chapter-structure}

\emph{Chapter Preview}. Begin with a chapter preview to set the stage of
the chapter. Do not finish with a ``preview of upcoming chapter'';
finish instead with a ``Further Reading and References.'' This consists
of a series of references with one or two lines of annotation for each
reference that the interested reader could follow up on (self-citations
are okay!). Historical developments are particularly nice in this
section.

This chapter begins with Section \ref{S:StructureBody} which provides a
general overview of a section's main body. Section \ref{S:SampleSection}
includes methods for referencing material in a chapter and ways to
include tables and figures not generated by R code. Section
\ref{S:Links} displays useful links regarding R Markdown, bookdown, best
practices for R code and online actuarial resources for reference.

\section{Structure of The Main Body}\label{S:StructureBody}

\begin{center}\rule{0.5\linewidth}{\linethickness}\end{center}

In this section, you learn how to:

\begin{itemize}
\tightlist
\item
  Determine what and what not to include in a chapter
\item
  Include a technical supplement if needed
\item
  Assess types of exercises and book resources that are appropriate for
  a chapter
\end{itemize}

\begin{center}\rule{0.5\linewidth}{\linethickness}\end{center}

Split the chapter into 4-7 sections; within each section, introduce 0-5
subsections. Do not develop a deeper hierarchy (e.g., a
``sub-subsection''). Use nonlinear aspects of the web. For example,
detailed mathematical developments go into a technical appendix or are
simply hidden unless the viewer really wants to see the details. Case
studies and historical references can be included in ``side-bars,'' a
supporting webpage. For the main body of the chapter, think about ``25
pages'' in length (whatever that means\ldots{}.).

\subsection{Main Body}\label{main-body}

\textbf{What to Include:}

\begin{itemize}
\tightlist
\item
  Within the chapter, use boxed and numbered lists of procedures for
  easy reference. It is certainly okay (and expected) to use
  mathematical notation. Each chapter should have examples interwoven
  within theory, allowing readers to see the development of the theory
  along with the importance of the applications.
\item
  Distinguish between an \emph{``Example''} and a \emph{``Special
  Case''}. The former shows how to relate the mathematics to a practical
  situation likely to be encountered by a practicing actuary. The latter
  looks at a subset of a general (usually) mathematical result. A few
  special cases are certainly acceptable but we want to focus on
  developing examples.
\item
  Think of graphical ways to visualize/summarize relationships that you
  want to emphasize.
\item
  Begin each section with a short bullet list describing the learning
  objectives of that section. Finish each section with a short quiz on
  these learning objectives. As of this writing (May 2016), quizzes are
  multiple-choice.
\item
  Include short exercises/examples/special cases that can be readily
  solved by the viewer (with solutions using ``hide/show'' features)
  within the main body. These serve to reinforce concepts and provide
  benchmarks for understanding.
\end{itemize}

\textbf{What Not to Include:} Do not include development of
equations/formulas in the main body of the text. The main body of the
text will be devoted to presenting results, providing context and
intuition as to the importance of the results. Do not include references
to the literature. This will appear in the last section on ``Further
Reading and References.'' Do not include graphs whose information could
easily be summarized by a table.

\subsection{Technical Supplements}\label{technical-supplements}

We want our viewers to understand the underpinnings of the theory (the
old analogy of ``what is going on under the hood to see how the engine
works'' - no black boxes.) So, there will be occasions when you feel
like a short development or ``proof''; is reasonable. Put this in an
appendix. Technical supplements should develop the theory in a
step-by-step fashion, building on each concept in a crisp, mathematical
fashion.

\subsection{Exercises}\label{exercises}

We anticipate that substantial exercise banks will be built over time by
users, professional associations, and those with commercial interests.
At this early stage of developing the chapter foundations, we recommend
developing the following types of exercises. Exercises will be segmented
by section (not subsection) and will be positioned at the end of each
chapter.

\begin{itemize}
\tightlist
\item
  \textbf{Hand Calculation.} Similar to those appearing within the
  chapter, include at the end of the chapter short
  exercises/examples/special cases that can be readily solved by the
  viewer.
\item
  \textbf{Software.} Include exercises that ask the viewer to work with
  ``R'' software, such as calculating a function or reproducing a graph.
\item
  \textbf{Data.} The need for working with real data is well documented;
  for example, see Hogg (1972), Moore and Roberts (1989) or Singer and
  Willett (1990). By providing detailed guided tutorials that work with
  theory and data, we teach our students the essence of Loss Data
  Analtyics. Of course, there are some important disadvantages to
  working with real data. Data sets can quickly become outdated.
  Further, the ideal data set to illustrate a specific statistical issue
  is difficult to find. Data exercises are complex and can span several
  chapter sections as well as chapters.
\end{itemize}

\subsection{Book Resources Supporting Each
Chapter}\label{book-resources-supporting-each-chapter}

There will be several resources support the book that will appear
outside of the chapter structure, including:

\begin{itemize}
\tightlist
\item
  \textbf{Case Studies and Historical Vignettes.} Similar to those
  appearing within the chapter, include short exercises/examples/special
  cases that can be readily solved by the viewer. These serve to
  reinforce concepts and provide benchmarks for understanding. Case
  studies can be used to emphasize different practices in different
  countries. Historical vignettes can be interesting in their own right
  and remind us all of the foundations of our discipline.
\item
  \textbf{Data.} We anticipate developing a library of data sets that
  can be used by instructors who wish to emphasize different areas of
  practice.
\item
  \textbf{Technical Supplements, Lists, and Tables.} The roles of
  technical supplements has already been described and there could be
  many. As is common in textbooks, we will also provide a place for
  lists or tables of organized facts for learners.
\end{itemize}

\subsection{Software Support}\label{software-support}

We will not focus on developing ``R'' tutorials but will provide guides
and links to people who wish to learn ``R''. Our focus is on teaching
statistical methods and actuarial issues, not software. We also will
provide support for users of other software environments, such as
Microsoft's Excel and SAS.

\section{Samples of Useful Elements}\label{S:SampleSection}

\begin{center}\rule{0.5\linewidth}{\linethickness}\end{center}

In this section, you learn how to:

\begin{itemize}
\tightlist
\item
  Reference other sections and equations
\item
  Include in-text citation that links to the bibliography
\item
  Include tables and figures not generated by R code
\item
  Include a footnote
\end{itemize}

\begin{center}\rule{0.5\linewidth}{\linethickness}\end{center}

\subsection{Section References}\label{section-references}

\begin{verbatim}
## Samples of Useful Elements {#S:SampleSection}
...

### Section References
... Section \\ref{S:SampleSection}.
\end{verbatim}

Here is some text to explain the current topic. It would be helpful if
you could recall a certain concept we covered previously in Section
\ref{S:SampleSection}.

\subsection{Equation References}\label{equation-references}

Here is an example of an equation using Latex in R Markdown.

\begin{verbatim}
\begin{equation}
x + y = 1  
\label{eq:ExampleEquation}
\end{equation}
\end{verbatim}

\begin{equation}
x + y = 1  
\label{eq:ExampleEquation}
\end{equation}

\begin{verbatim}
... equation \\eqref{eq:ExampleEquation}
\end{verbatim}

Instead of writing from the equation above, we can use from equation
\eqref{eq:ExampleEquation} which is linked to the equation itself.

\subsection{In-text Citations}\label{in-text-citations}

\begin{verbatim}
... R Bookdown [@xie2015]
\end{verbatim}

Here is an example of an in-text citation made possible by R Bookdown
\citep{xie2015}. This links to the bibliography where the full reference
is displayed. As a convention we use APA style citation.

\subsection{Including Tables}\label{including-tables}

In order to include table not generated by R such as a Latex table, we
have to make some adjustments to regular Latex syntax.

\begin{verbatim}
$$\begin{matrix}
\begin{array}{|c|c|} \hline
\text{Policyholder} & \text{Number of claims} \\\hline
\textbf{X} & 1 \\\hline
\textbf{Y} & 2 \\\hline
\end{array}
\end{matrix}$$
\end{verbatim}

\[\begin{matrix}
\begin{array}{|c|c|} \hline
\text{Policyholder} & \text{Number of claims} \\\hline
\textbf{X} & 1 \\\hline
\textbf{Y} & 2 \\\hline
\end{array}
\end{matrix}\]

Table 1: An example of including tables using Latex in an R Markdown
document

R Markdown does not have a convention for referencing non-R generated
tables. For now, we reference them manually as in ``refer to Table 1''.

\subsection{Including Figures}\label{including-figures}

For figures, we have yet to find a method to include non-R generated
figures in modifiable code form as seen with the tables. As a temporary
solution, we store the figures as png or jpeg files in a separate folder
called ``Figures''. Then we use R code to call those figures for display
so that we can reference them.

\begin{Shaded}
\begin{Highlighting}[]
\NormalTok{knitr}\OperatorTok{::}\KeywordTok{include_graphics}\NormalTok{(}\StringTok{"Figures/RStudio-Ball.png"}\NormalTok{)}
\end{Highlighting}
\end{Shaded}

\begin{figure}

{\centering \includegraphics[width=0.05\linewidth]{Figures/RStudio-Ball} 

}

\caption{An example of including figures in an R Markdown document}\label{fig:ExampleFigure}
\end{figure}

\begin{verbatim}
... Figure \\ref{fig:ExampleFigure}
\end{verbatim}

As you can see from Figure \ref{fig:ExampleFigure}, we now know how to
include non-R generated figures in our R Markdown document.

\subsection{Including Footnotes}\label{including-footnotes}

\begin{verbatim}
... [^1]

[^1]: ... # note: the footnote displays at the end of the document
\end{verbatim}

Here is how you can include a footnote \footnote{A footnote.}.

\section{Useful Links}\label{S:Links}

\begin{center}\rule{0.5\linewidth}{\linethickness}\end{center}

In this section, you learn how to:

\begin{itemize}
\tightlist
\item
  Use R Markdown and bookdown
\item
  Style R code according to best practices
\item
  Use online actuarial text resources as examples
\end{itemize}

\begin{center}\rule{0.5\linewidth}{\linethickness}\end{center}

For an R Markdown guide refer
\href{https://rmarkdown.rstudio.com/authoring_pandoc_markdown.html}{here}.

For a bookdown guide refer
\href{https://bookdown.org/yihui/bookdown/}{here}.

For R best practices refer
\href{http://r-pkgs.had.co.nz/style.html}{here}.

For online actuarial text resources refer
\href{https://sites.google.com/a/wisc.edu/loss-data-analytics/online-actuarial-text-resources}{here}.

\section{Exercises}\label{exercises-1}

Here are a set of exercises at the end of the chapter.

\section{Contributors and Further
Resources}\label{further-reading-and-resources}

\subsubsection*{Contributor}\label{contributor}
\addcontentsline{toc}{subsubsection}{Contributor}

\begin{itemize}
\tightlist
\item
  \textbf{Edward W. (Jed) Frees}, University of Wisconsin-Madison
\end{itemize}

\bibliography{book.bib,packages.bib}


\end{document}
