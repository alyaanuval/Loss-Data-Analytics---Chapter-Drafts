\documentclass[12pt]{article}
\usepackage{makeidx}
\usepackage{multirow}
\usepackage{multicol}
\usepackage[dvipsnames,svgnames,table]{xcolor}
\usepackage{graphicx}
\usepackage{epstopdf}
\usepackage{ulem}
\usepackage[bookmarks]{hyperref}
\usepackage{amsmath}
\usepackage{amsfonts}
\usepackage{amssymb}
\usepackage{graphics}
\usepackage{epsfig}
\usepackage{verbatim}
% new LaTeX commands
\usepackage{geometry}
\geometry{left=1in,right=1in,top=1in,bottom=1in}
\usepackage{framed}    % to make framed boxes
\usepackage{scalefnt}
\usepackage{alltt}
\usepackage{hyperref}
\hypersetup{
    bookmarks=true,         % show bookmarks bar?
    unicode=false,          % non-Latin characters in Acrobat's bookmarks
    pdftoolbar=true,        % show Acrobat's toolbar?
    pdfmenubar=true,        % show Acrobat's menu?
    pdffitwindow=true,      % page fit to window when opened
    pdftitle={My title},    % title
    pdfauthor={Author},     % author
    pdfsubject={Subject},   % subject of the document
    pdfnewwindow=true,      % links in new window
    pdfkeywords={keywords}, % list of keywords
    colorlinks=true,        % false: boxed links; true: colored links
    linkcolor=blue,         % color of internal links
    citecolor=blue,        % color of links to bibliography
    filecolor=magenta,      % color of file links
    urlcolor=cyan           % color of external links
}
\newcommand{\type}[1]{{\tt$\backslash$#1}}

%Personal definitions - Bayes Regression
\def\bsb{\boldsymbol \beta}
\def\bsa{\boldsymbol \alpha}
\def\bsm{\boldsymbol \mu}
\def\bsS{\boldsymbol \Sigma}
\def\bsx{\boldsymbol \xi}
\def\E{\mathrm{E}}
\def\Var{\mathrm{Var}}
\def\Cov{\mathrm{Cov}}

\newenvironment{my_enumerate}{
\begin{enumerate}
  \setlength{\itemsep}{1pt}
  \setlength{\parskip}{0pt}
  \setlength{\parsep}{0pt}}{\end{enumerate}
}
\newenvironment{my_itemize}{
\begin{itemize}
  \setlength{\itemsep}{1pt}
  \setlength{\parskip}{0pt}
  \setlength{\parsep}{0pt}}{\end{itemize}
}

\usepackage{natbib}
%\bibliographystyle{asa}
%\bibliographystyle{chicago}
\bibliographystyle{econPeriod}

\begin{document}
%\begin{flushright}{{\tiny \textit{\today} }}\end{flushright}

\title{Ch 8. Risk classification}
\author{Contributing Author: Joseph H.T. Kim, \textit{Yonsei University}
}

\date{\today}
\maketitle

\textit{Chapter Preview.} To be done.

%Introduction to risk classification using Poisson regression models
%Exposure to risk
%Categorical variables and multiplicative tariff
%Extensions to generalized linear models
%Technical Supplement 3.  Likelihood and generalized linear models

%================================
\section{Introduction}
Through insurance contracts, the policyholders effectively transfer these risks to the insurer in exchange for premiums. For the insurer to stay in business, the premium income collected from a pool of policyholders must at least equal to the benefit outgo. Ignoring the frictional expenses associated with the administrative cost and the profit margin, the net premium should be equal to the expected loss occurring from the risk that is transferred from the policyholder to the insurer. 

If all policyholders in the insurance pool have identical risk profiles, the insurer may simply charge the same premium for each policyholder because all of them would have the same expected loss. In reality however the policyholders are hardly homogeneous. For example, mortality risk in life insurance depends on the characteristics of the policyholder, such as, age, sex and life style. In auto insurance, those characteristics may include age, occupation, the type or use of the car, and the area where the driver resides. The knowledge of these characteristics or variables of individual policyholders can enhance the ability of calculating a fair premium as they can be used to estimate or predict the expected losses more accurately at the individual level. 

However, this individual pricing would be impractical in reality because identifying, measuring and incorporating all relevant risk characteristics are quite cumbersome, difficult and expensive for insurers; some relevant risk characteristics, e.g., driving skills, are not even directly observable. Instead, insurers maintain some classification system which assigns each policyholder to one of the risk classes based on a relatively small number of risk characteristics that are deemed more relevant. These characteristics used in the classification system are called the rating factors; these factors are \textit{a priori} variables in the sense that they are known before the contract begins. Therefore, a risk classification system can be thought of as a compromise between the two extreme premiums: the true individual premium and the collective premium where all policyholders pay the same premium regardless of their risk characteristics. All policyholders in the same risk class therefore pay the same premium as they share identical risk factors.

An important task in this risk classification is to construct a quantitative model that can determine the expected loss given various rating factors for a policyholder. The standard approach is to adopt a statistical regression model which produces the expected loss as the output when the relevant risk factors are given as the inputs. We introduce and discuss the Poisson regression, which can be used when the loss is a count variable, as a prominent example of an insurance pricing tool under a risk classification scheme.

%\subsection{Fundamental concept of insurance pricing}
%================================
\section{Poisson regression model}
%================================
The Poisson regression model has been successfully used in a wide range of applications and has an advantage of allowing closed-form expressions for the important quantities, which could provide a better intuition. We start our discussion with the standard linear regression model.
\subsection{Linear regression}
Suppose that we have a single a priori rating factor denoted $x$ for each policyholder along with the past loss amount, which is also a single number $y$. Then for a pool of $n$ independent policyholders the insurer's dataset consists of $(x_i, y_i)$, $i=1, \ldots,n$, where the subscript indicates the $i$th policyholder. The simplest linear regression model then postulates a linear relation between $x$ and $y$, so that 
\begin{equation}
\label{simple.linear.reg}
y_i=\beta_0+\beta_1x_i+\epsilon_i, \quad i=1, \ldots,n.
\end{equation}
Here $\beta_0$ and $\beta_1$ are the common intercept and slope coefficients which apply to all policyholders. The last term $\epsilon_i$, known as the random error term, represents the random uncertainty that cannot be captured by the pure deterministic equation $y=\beta_0+\beta_1x$.  It is commonly assumed that $\epsilon_i \sim_{iid} N(0, \sigma^2)$, so that we have, from (\ref{simple.linear.reg}),
\begin{equation}
\label{simple.linear.reg.ft}
\E(y_i|x_i)=\beta_0+\beta_1x_i.
\end{equation} A generic form of this model without the subscripts, called the regression or mean response function, is written as
\begin{equation}
\label{simple.linear.reg.ft2}
\E(y|x)=\beta_0+\beta_1x
\end{equation}
Thus the \textit{expected} loss $\E(y|x)$ is a linear function of the rating factor $x$. In the statistical literature, the input $x$ is often referred to as the predictor or explanatory variable, and the output $y$ as the response variable. This functional form indicates that when the rating factor $x$ increases by 1, the loss amount increases by $\beta_1$ on average. This formulation however is merely a theoretical one because the true values of $\beta_0$ and $ \beta_0$ are unknown. To obtain a workable model, we estimate these parameters using, e.g., maximum likelihood estimation (mle)\footnote{For the linear regression, the mle and ols yield the same estimators. } based on the normal distributional assumption for $\epsilon_i$. By denoting these estimates by $\hat{\beta}_0=b_0$ and $\hat{\beta}_1=b_1$, we obtain the estimated regression function 
\begin{equation}
\label{y.hat.simple.reg}
\hat{y}_i=b_0+b_1x_i,
\end{equation} paralleling Equation (\ref{simple.linear.reg.ft}). Thus, if the postulated linear model in (\ref{simple.linear.reg}) is accepted, we can estimate the expected loss for the $i$th policyholders simply as $\hat{y}_i=b_0+b_1x_i$, and policyholders with a different rating factor produces a different premium.

When there are more than one rating factors, as is the case in practice, we extend this simple linear form (\ref{simple.linear.reg}) into a multiple linear regression. When there are $k \ge 1$ different rating factors for each policyholder, the record of the $i$th policyholder would consist of $(x_{i1}, x_{i2}, \ldots , x_{ik}, y_i)$, $i=1, \ldots,n$, and the multiple linear regression postulates
\begin{equation}
\label{mult.linear.reg}
y_i=\beta_0+\beta_1x_{i1}+\beta_2x_{i2}+\ldots + \beta_k x_{ik}+\epsilon_i, \quad i=1, \ldots,n.
\end{equation}
By using matrix notation we can compactly rewrite this model as
\begin{equation}
\label{mult.linear.reg.mtx}
\mathbf{ y}=\mathbf{ X\beta} +\mathbf{ \epsilon}
\end{equation}
where each vector and matrix are given by
 \begin{eqnarray*}
% Y &=& X \beta + \epsilon \\
% \noalign{\hbox{where}}
\mathbf{ y} &=& \left( \begin{array}{c} y_1 \\ y_2 \\ \vdots \\ y_n
              \end{array} \right), \qquad
{\beta} = \left( \begin{array}{c} \beta_0 \\ \beta_1 \\ \vdots \\ \beta_{k}
              \end{array} \right), \qquad
{\epsilon} = \left( \begin{array}{c} \epsilon_1 \\ \epsilon_2 \\ \vdots \\ \epsilon_n
              \end{array} \right), \qquad \text{and} \quad
\mathbf{ X} = \left( \begin{array}{cccc}
        1 & x_{11} & ... &x_{1k} \\
        1 & x_{2}  & ... &x_{2k}\\
       \vdots & \vdots  & \ddots  & \vdots \\
  %      1 & X_{i1} & X_{i2} & \ldots & X_{ip} \\
  %     \vdots & \vdots & \vdots & \vdots  \\
        1 & x_{n} & ... &x_{n,k} \end{array} \right)  \\
 \end{eqnarray*} 
The corresponding regression function is then given by
\begin{equation}
\label{mult.linear.reg.ft3}
\E({ y}|\mathbf{ x})=\mathbf{ x^{\prime}\beta}
\end{equation} where $\mathbf{ x}=(1, x_1, \ldots, x_k)^{\prime}$ is the vector of some hypothesized rating factors. As before, the parameter vector $\beta$ must be estimated from the dataset, and one can obtain the estimated regression function 
\begin{equation}
\label{mult.linear.reg.ft4}
\hat{ y}= b_0+b_1 x_1+ \ldots +b_k x_k= \mathbf{ x^{\prime}b},
\end{equation}
where  $\mathbf{ b}=(b_0, \ldots, b_k)^{\prime}=(\hat{\beta}_0, \ldots, \hat{\beta}_k)^{\prime}$. 

The linear regression models are important statistical tools in data analyses and encompass a wide variety of extensions but we do not delve into this too much as our goal here in this chapter is to introduce an alternative regression model that can handle count response variables.

%------------------------
\subsection{Need of Possion regression model} 
Now suppose that we have a dataset where the response variable is the loss count instead of the loss amount. For example, $y$ can be the number of accidents of a policyholder during a year. In this case the loss can take only non-negative integer values, i.e., $y_i=0,1,2, \ldots$ for each policyholder. Unfortunately, the multiple linear regression above, despite being popular and useful in many applications, cannot handle this 
type of dataset because of the following reasons. First, there is no guarantee that the output $\hat{y}$ in (\ref{y.hat.simple.reg}) is an integer value. It is easy to imagine that it can be a fractional number by adjusting some rating factors. Second, more importantly, there is possibility that $\hat{y}$ becomes negative for a peculiar set of rating factors. These unintended and undesirable consequences are actually due to the fact that the output of multiple linear regression models spans the whole real line, that is, $\hat{y}$ can take any real value in $(-\infty, \infty)$ by construction. More fundamentally, this stems from the very assumption of the normal error terms in the regression model itself. This can be readily verified by observing that, when  $\epsilon_i \sim_{iid} N(0, \sigma^2)$, the response variable is also normally distributed  according to (\ref{simple.linear.reg}):
\begin{equation}
\label{ }
y_i \sim_{iid} N(\beta_0+\beta_1x_i, \sigma^2),
\end{equation}
which confirms that the loss amount $y$ is allowed to take any real value.

One may impose restrictions on the linear regression so that it can take only non-negative integers as the response variable. Instead of seeking such ad hoc solutions however we consider the Poisson regression, a different type of regression model where the underlying random component is consistent with the count variable. We have encountered the Poisson distribution and studied its distributional properties  in early chapters. The pmf of the Poisson with mean $\mu$ is
\begin{equation}
\label{Pois.pmf}
\Pr(Y=y)=\frac{\mu^y e^{-\mu}}{y!},\qquad y=0,1,2, \ldots
\end{equation} and $\E(Y)=\Var(Y)=\mu$. 
Now suppose $Y$ stands for the random loss count, and it depends on $k$ rating factors $\mathbf{ x}=(1, x_1, \ldots, x_k)$, as before. Our goal is then to formulate a regression model which uses the Poisson distribution as the response variable and, at the same time, responds to risk factors of individual policyholders.  
%
%two abstraction:
%\\
%1. matrix representation\\
%2. below\\
%(see project webpage to see the notation and basic regression backgrounds)
%
%
%Actually we can see Equation (\ref{simple.linear.reg}) as a special case of the following general form
%\begin{equation}
%\label{ }
%y_i=f(x_i)+\epsilon_i
%\end{equation}
%By adopting different functions $f$, one can build a wide range of different models, though the validity and suitability of those models needs to be statistically addressed. 
%
%To fix the notation we assume:
%

\subsection{Formulaing Poisson regression}
As before we assume that $n$ independent policyholders with rating factors $\mathbf{ x}_i=(1, x_{i1}, \ldots, x_{ik})$ have recorded the loss count $y_i=0,1,2, \ldots $, for $i=1, \ldots, n$. We also assume, for now, that all policyholders have the same one unit period for loss observation, or equal exposure\footnote{We will discuss more details on the exposure in the following subsection.} of 1, to make things simpler. 
We start with looking at the linear regression model from a different angle. The linear regression formulated in (\ref{simple.linear.reg}) and  (\ref{simple.linear.reg.ft2}), when combined,  suggests an alternative description of the model
\begin{equation}
\label{ }
y_i= \E(y_i|\mathbf{ x}_i)+\epsilon_i
\end{equation} with $\E(y_i|\mathbf{ x}_i)=\mathbf{ x}^{\prime}_i\beta$, where $\beta=(\beta_0, \beta_1, \ldots, \beta_k)^{\prime}$. This expression tells that the response variable is the mean response function with the error term added. Along similar lines, we can describe the Poisson regression through its mean function. For this we first denote $\mu_i$ to be the expected loss count of the $i$th policyholder under the Poisson specification (\ref{Pois.pmf}): 
\begin{equation}
\label{ }
\mu_i=\E(y_i|\mathbf{ x}_i), \qquad y_i \sim Pois(\mu_i), \, i=1, \ldots, n.
\end{equation} 
It is noted here that in principle the conditional mean $\E(y_i|\mathbf{ x}_i)$ can take different forms depending on how we specify the relationship between $\mathbf{ x}$ and $y$, though it was chosen as a linear function in the linear regression. The standard choice for the Poisson regression is to adopt the exponential function so that
\begin{equation}
\label{mean.ft.Pois}
\mu_i=\E(y_i|\mathbf{ x}_i)=e^{\mathbf{ x}^{\prime}_i\beta}, \qquad y_i \sim Pois(\mu_i), \, i=1, \ldots, n.
\end{equation} 
Note that this specification resolves the limitations of the linear regression model previously mentioned. The exponential function ensures that $\mu_i >0$ for any set of rating factors $\mathbf{ x}_i$ and each $y_i$ appropriately forms count data generated from a non-negative discrete distribution.
Often (\ref{mean.ft.Pois}) is rewritten as
\begin{equation}
\label{mean.ft.Pois2}
\log \mu_i=\log \E(y_i|\mathbf{ x}_i)=\mathbf{ x}^{\prime}_i\beta, \qquad y_i \sim Pois(\mu_i), \, i=1, \ldots, n
\end{equation} to reveal the relationship when the right side is set as the linear form, $\mathbf{ x}_i\beta$. Again, we see that the  mapping works well as both sides of (\ref{mean.ft.Pois2}), $\log \mu_i$ and $\mathbf{ x}_i\beta$, can cover the entire real values. 
This is the formulation of the Poisson regression, assuming that all policyholders have the same unit period of exposure. When the exposures differ among the policyholders, however, as is the case in most practical cases, we need to revise this formulation by adding  exposure component.

%===================
\subsection{Incorporating exposure}
%===================
The assumption of the same one unit of exposure across all policyholders is not a realistic one. In auto insurance, two different drivers with different lengths of insurance coverage (e.g., 3 months and 12 months, respectively) could have recorded the same number of accidents. As the expected number of accidents would be proportional to the length of coverage, we should not treat these two drivers' loss experiences identically in the modelling process. This motivates the need of the concept of the exposure in the Poisson regression. 

The Poisson distribution in (\ref{Pois.pmf}) is parametrised via its mean. To understand the exposure, we alternatively parametrize the Poisson pmf in terms of the \textit{rate} parameter $\lambda$, inspired from the definition of the Poisson process:
\begin{equation}
\label{Pois.pmf.2}
\Pr(Y=y)=\frac{(\lambda t)^y e^{-\lambda t}}{y!},\qquad y=0,1,2, \ldots
\end{equation} with $\E(Y)=\Var(Y)=\lambda t$. Here $\lambda$ is understood as the rate or intensity per unit period of the Poisson process and $t$ represents the length of time or \textit{exposure}, a known constant. As the exposure $t$ gets larger, the Poisson distribution produces a larger expected loss count as it should. 
Clearly, (\ref{Pois.pmf.2}) reduces to (\ref{Pois.pmf}) when $t=1$, which means that the mean and the rate become the same for the unit  exposure, the case we considered in the previous subsection. In principle the exposure does not need to be measured in units of time and may represent different things depending the problem at hand. For example,
\begin{enumerate}
  \item  In health insurance, the rate may be the occurrence of a specific disease per 1,000 people and the exposure is the number of people considered in the unit of 1,000.
    \item  In auto insurance, the rate may be the number of accidents per year of a driver and the exposure is the length of the observed period for the driver in the unit of year.

  \item In marketing, the rate may be  the number of customers who enter a store per hour and the exposure is the number of hours observed.
  \item In civil engineering, the rate may be  the number of major cracks on the paved road per 10 kms and the exposure is the length of road considered in the unit of 10 kms.
  \item In biology, the rate may be  the number of a specific type of bacteria found per 1 cm$^3$ of sea water and the exposure is the volume of sea water considered in the unit of cubic centimeter. 
\end{enumerate}
Incorporating the exposure in the Poisson regression thus requires us to carefully separate the rate and exposure in the modelling process. Focusing on our insurance context, if we denote the rate of the loss event of the $i$th policyholder by $\lambda_i$, the known exposure (the length of coverage) by  $m_i$ and the expected loss count under the given exposure by $\mu_i$, then  the Poisson regression formulation in (\ref{mean.ft.Pois}) and (\ref{mean.ft.Pois2}) 
should be revised in light of (\ref{Pois.pmf.2}) as
\begin{equation}
\label{mean.ft.Pois6}
\mu_i=\E(y_i|\mathbf{ x}_i)=m_i \,\lambda_i=m_i \, e^{\mathbf{ x}^{\prime}_i\beta}, \qquad y_i \sim Pois(\mu_i), \, i=1, \ldots, n,
\end{equation} 
 and 
\begin{equation}
\label{mean.ft.Pois7}
\log \mu_i=\log m_i+\mathbf{ x}^{\prime}_i\beta, \qquad y_i \sim Pois(\mu_i), \, i=1, \ldots, n.
\end{equation}
Adding $\log m_i$ in (\ref{mean.ft.Pois7}) does not pose a problem in fitting as we can always specify this as an extra explanatory variable, as it is a known constant, and fix its coefficient to 1. In the literature $\log m_i$ is commonly called the \textit{offset}.

%===================
\subsection{Estimating Poisson regression model}
%===================

%===================
\subsection{Exercises}
%===================
\begin{enumerate}
  \item Show that the mle and ols are the same for the simple linear regression.  
  \item In the Poisson regression formulation, consider using $\mu_i=\E(y_i|\mathbf{ x}_i)=({\mathbf{ x}^{\prime}_i\beta})^2$, for  $i=1, \ldots, n$,  instead of the exponential function  What potential issue would you have?
  \item 
\end{enumerate}
%===================
\section{Categorical variables and multiplicative tariff}
%===================
%In this section (see other authors work of ratios)
%\begin{itemize}
%  \item 
%  \item 
%  \item 
%\end{itemize}
%===================
\subsection{Rating factors and tariff}
%===================
So far we have implicitly assumed that the risk characteristics of individual policyholders can be considerably diverse. However, this may not be so because in practice most rating factors in insurance are categorical variables, meaning that they take one of the pre-determined number of possible values. Examples of categorical variables include sex, type of cars, the driver's region of residence and occupation. Continuous variables, such as age or auto mileage, can also be grouped by bands and treated as categorical variables. Thus for a large insurer with only a few rating factors, there will be many policyholders in each risk class. For the remaining of this chapter we assume that all rating factors are categorical variables. 

To further elaborate how rating factors are used, we consider a hypothetical auto insurer with only two rating factors:
\begin{itemize}
  \item Type of vehicle: Type A (personally owned) and B (owned by corporations). We use index $j=1$ and $2$ for this rating variable.
  \item Age band of the main driver:  Young (age $<$ 25), middle (25 $\le$ age $<$ 60) and old age (age $\ge$ 60). We use index $k=1, 2$ and $3$, respectively, for this rating variable.
%  \item 
\end{itemize}
Thus there are $2 \times 3=6$ different risk classes in total for this insurer. From these classification rule, we may create an organized  table or list called \textit{tariff}, such as the one shown in Table \ref{tab.tariff.1}, collected from all policyholders. These six different risk classes are called tariff cells. Each tariff cell represents a unique combination of the rating factors, and all policyholders in the same tariff cell are considered to be homogeneous and charged the same premium.
\begin{table}[htp]
\caption{Tariff of an illustrative auto insurer}
\begin{center}
\begin{tabular}{rrrrc}
 \hline
 \multicolumn{2}{c}{Rating factors } &  Risk class ($i$)& Exposure & Claim count \\
 \cline{1-2} 
Type ($j$) & Age ($k$) & (Tariff cell) &  in year & observed\\
\hline \hline
$j=$1 & $k=$1 & $i=$1 & 89.1 & 9\\
1 & 2 &2  & 208.5& 8\\
1 & 3 & 3 & 155.2 & 6  \\
2  & 1 & 4 & 19.3 & 1 \\
2  & 2 & 5 & 360.4 & 13 \\
2   & 3 & 6 & 276.7 & 6 \\ \hline
\end{tabular}
\end{center}
\label{tab.tariff.1}
\end{table}%

In the table the exposure means the sum of the length of insurance coverages, or in-force times, in the unit of year, of all the policyholders in each tariff cell. Similarly each claim count in the table is the number of claims at each cell. Clearly the exposures and claim counts vary due to the different number of drivers across the cells, as well as different in-force time periods among the drivers within each cell.

Now, we denote the exposure and claim count of cell $(j,k)$ as $m_{jk}$ and $y_{jk}$, respectively, and define the claim count per unit exposure as
\begin{equation}
\label{ }
z_{jk}= \frac{y_{jk}}{ m_{jk}}, \qquad j=1,2;\, k=1, 2,3.
\end{equation}
For example, $z_{12}=8/208.5=0.03837$, meaning that a policyholder in tariff cell (1,2) would have 0.03837 accidents if insured for a full year on average. The set of $z_{ij}$ values then corresponds to the rate parameter in the Poisson distribution (\ref{Pois.pmf.2}) as they are the event occurrence rates per unit exposure in the Poisson precess. We also note that the index pair $(j,k)$ can be replaced by a single index $i$ by mapping two risk factors to risk class as shown in the third column of the table, which sometimes can be convenient in our discussion. If there are many risk factors, the number of cells grows exponentially and the tariff would consist of a set of tables, instead of one, separated by some of the basic rating factors, such as sex or territory. 

In order to apply the Poisson regression model for the tariff in Table \ref{tab.tariff.1}, we need to convert the categorial rating factors   into numerical values. This is the topic of the next subsection.


%===================
\subsection{Multiplicative tariff model}
%===================
We will assume the same illustrative auto insurer with two rating factors, car type and age. Recall that the loss count or frequency of a policyholder is described by the Poisson regression model with rate $\lambda$ per unit exposure and the amount of exposure $m$, so that the expected loss count becomes $m\lambda$. As $\lambda$ should respond to the change of the rating factors, 
we need to relate it to the rating factors in some functional form. Among others, we consider the following multiplicative functional relation, which is common in practice, 
\begin{equation}
\label{multiplicative.tarrif}
\lambda_{jk}= f_0 \times f_{1j} \times f_{2k}, \qquad j=1,2;\, k=1, 2,3.
\end{equation} 
Here $\{ f_{1j}, j=1,2\}$ are the parameters associated with the two levels in the first rating factor, car type, and  $\{ f_{2k}, k=1,2,3\}$ are associated with the three levels in the age band, the second rating factor. $f_0$ is some base value to be discussed later. Thus these parameters are the numerical counterparts of the categorial rating factors, and are to be estimated from the dataset.  The  model (\ref{multiplicative.tarrif}) explains how the expected loss count (per unit exposure) changes as each rating factor varies. For example, if $f_{11}=1$ and $f_{12}=1.2$, then the expected loss count of a policyholder with vehicle type B would be 20\% larger than the one with type B, when other factors are fixed. However, (\ref{multiplicative.tarrif}) has an identification issue because, for any $c>0$, we have 
\begin{equation}
\label{multi.tariff.2}
\lambda_{ij}= f_0 \times \frac{f_{1j}}{c} \times c\,f_{2k}, 
\end{equation} indicating that the rate parameter can be identical for very different individual rating factors. This over-parametrization, meaning that many different sets of parameters arrive at the identical model, obviously calls for some restriction on the rating factors. The standard practice is to make one parameter in each rating factor equal to one. This can be made arbitrarily, so we assume that $f_{11}=1$ and $f_{21}=1$ for our illustration. This way all other parameters are uniquely determined. The tariff cell $(1,1)$ is then called the base cell, in which the rate is simply $\lambda_{11}=f_0$, the base value, according to (\ref{multiplicative.tarrif}). Thus we generally interpret $f_0$ as the base value that is equal to the Poisson rate of the base cell. In non-life insurance, the parameters $f_{1j}$ and $f_{2k}$  are known as price relatives as they determine how much expected loss, the net premium, changes relative to the base value $f_0$. Often (\ref{multiplicative.tarrif}) is log-transformed and rewritten as
\begin{equation}
\label{log-linear.tariff}
\log \lambda_{jk}= \log f_0 + \log f_{1j} + \log f_{2k}, 
\end{equation} as it is easier to work with a linear form in estimating process. It is also advantageous as log intensity in the left side can take any real value with $\lambda_{jk}>0$, similar to (\ref{mean.ft.Pois2}).

The log-linear model (\ref{log-linear.tariff}) resembles the two-way analysis of variance (ANOVA) in Statistics, which studies how two different factors or treatments affect the mean outcome of quantity of interest. Thus we may adopt the standard statistical techniques for our tariff analysis. 

%One may also consider the additive form, i.e., $\lambda_{ij}= f_0 + f_{1i} + f_{2j}$, but the multiplicative version is known to be more appropriate for actual applications. By taking logarithm on (\ref{multi.tariff.1}) we have the log-linear form
%\begin{equation}
%\label{multi.tariff.4}
%\log \lambda_{ij}= \log f_0 + \log f_{1i} + \log f_{2j}, \qquad i=1,2; j=1, \ldots, 5
%\end{equation}

%===================
\subsection{Poisson regression for multiplicative tariff}
%===================
We now turn to the estimation of these price relatives. For this we need to know how categorial variables can be incorporated in regression models. When a categorial variable is used as an explanatory variables in a regression model, it is customary to use an indicator variable which takes either 0 or 1. For our illustrative insurer, we may define for the first rating variable as
\begin{equation}
\label{ind.x1}x_1=
\begin{cases}
      1 & \text{ for vehicle type B}, \\
      0 & \text{ for vehicle type A}.
\end{cases}
\end{equation} 
For the second rating variable, however, an indicator variable does not work directly as there are three possible levels. The trick is to use two indicator variables for the age band\footnote{If there are $l$ levels for a given rating factor we need $l-1$ indicator variables.}, that is,
\begin{equation}
\label{ind.x2}x_2=
\begin{cases}
     1 & \text{for age band 2}, \\
     0 & \text{otherwise}.
\end{cases}
\end{equation} and 
\begin{equation}
\label{ind.x3}x_3=
\begin{cases}
     1 & \text{for age band 3}, \\
     0 & \text{otherwise}.
\end{cases}
\end{equation} The triple $(x_1, x_2, x_3)$ then can effectively and uniquely determine each risk class. For example, cell $(j,k)=(2,2)$ with vehicle type B and age band 2, corresponds to $(1, 1, 0)$, and the base cell $(j,k)=(1,1)$ corresponds to $(0, 0, 0)$.
Now we can express the log intensity (\ref{log-linear.tariff}) in terms of indicator variables as  
\begin{equation}
\label{log-linear.tariff.3}
\log \lambda_{jk}= \log f_0+ \log f_{12}  \times x_1 + \log f_{22} \times x_2 +\log f_{23} \times x_3.
\end{equation} For example, for a policyholder at cell $(j,k)=(2,2)$ we have $(x_1, x_2,x_3)=(1,1,0)$ to yield, from (\ref{log-linear.tariff.3}),  $\log \lambda_{22}=\log f_0+\log f_{22}$, and for the  base cell $(j,k)=(1,1)$ corresponding to $(x_1, x_2,x_3)=(0, 0, 0)$, we obtain $\log \lambda_{11}=\log f_0$.

It is important to note that we have carefully selected the coefficients for each indicator variable in (\ref{log-linear.tariff.3}), so that the base cell produces a vector of zeros, $(x_1, x_2,x_3)=(0, 0, 0)$. This way, the log intensity for the base cell yields $\lambda_{11}=f_0$ as required. Also note that we do not need to include $f_{11}$ and $f_{21}$ in (\ref{log-linear.tariff.3}) as $\log f_{11}=\log f_{21}=0$ by construction. If we reverse the binary switch for $x_i$ in  (\ref{ind.x1}) -- (\ref{ind.x3}) or select different coefficients in (\ref{log-linear.tariff.3}), we will lose this internal consistency;
% so that it is set 1 for vehicle type A and 0 for type B, then the base cell $(j,k)=(1,1)$ produces $(x_1, x_2,x_3)=(1, 0, 0)$, which in turn gives $\log \lambda_{11}=\log f_0+ \log f_{12}$ or $\lambda_{11}=f_0\,f_{12}$ from (\ref{log-linear.tariff.3}). This is not acceptable as $f_0\,f_{12}$ describes the price relativity of cell $(2, 1)$ instead of $(1, 1)$, violating our requirement of $\lambda_{11}=f_0$. 
in an exercise question, the reader is invited to check this. 

Under this specification, let us convert the tariff cell index $(j,k)$ to a single risk class index $i$ as shown in Table \ref{tab.tariff.1}. Then for the $i$th risk class with triple $(x_{i1}, x_{i2},x_{i3})$, we now recognize  (\ref{log-linear.tariff.3}) as
\begin{equation}
\label{log-linear.tariff.4}
\log \lambda_{i}= \beta_0+ \beta_1 \, x_{i1} + \beta_{2} \, x_{i2} +\beta_3  \, x_{i3}=\mathbf{ x}^{\prime}_i\beta, \qquad i=1, \ldots, 6,
\end{equation} where $\beta_0, \ldots, \beta_3$ can be found in the corresponding position in (\ref{log-linear.tariff.3}). This in fact is exactly what we did in (\ref{mean.ft.Pois2}). Therefore, by incorporating the exposure in each risk class we can readily estimate parameters $\beta_0, \ldots, \beta_3$ and thus $f_{1j}$ and $f_{2k}$ in the same way as we did in (\ref{mean.ft.Pois6}) and (\ref{mean.ft.Pois7}). In Table \ref{tab.tariff.2} we present the mapping between $\beta_i$ coefficients and the price relatives, keeping the same list form as Table \ref{tab.tariff.1}. All columns except for the first is indexed with $i$, giving us a complete specification for our Poisson regression.

\begin{table}[htp]
\caption{Tariff parametrization of an illustrative auto insurer}
\begin{center}
\begin{tabular}{cccccc}
 \hline
{Rating factors } &  Risk class & Covariates & Price relativity & Exposure & Claim count \\
$(j,k)$ & $i$ & $(x_{i1}, x_{i2},x_{i3})$ & $\lambda_i=e^{\mathbf{ x}_i\beta}$&  $m_i$ & $y_i$\\
\hline \hline
$(1,1)$ &  1 & $(0,0,0)$ & $\lambda_1=f_0f_{11}f_{21}$ & 89.1 & 9\\
$(1,2)$ &2  & $(0,1,0)$ & $\lambda_2=f_0f_{11}f_{22}$ & 208.5& 8\\
$(1, 3)$ & 3 &$(0,0,1)$ & $\lambda_3=f_0f_{11}f_{23}$ & 155.2 & 6  \\
$(2 , 1)$ & 4 & $(1,0,0)$ & $\lambda_4=f_0f_{12}f_{21}$ & 19.3 & 1 \\
$(2 , 2)$ & 5 & $(1,1,0)$ & $\lambda_5=f_0f_{12}f_{22}$& 360.4 & 13 \\
$(2 , 3)$ & 6 & $(1,0,1)$ & $\lambda_6=f_0f_{12}f_{23}$ & 276.7 & 6 \\ \hline
\end{tabular}
\end{center}
\label{tab.tariff.2}
\end{table}%
% 
%\begin{table}[htp]
%\caption{Tariff parametrization of an illustrative auto insurer}
%\begin{center}
%\begin{tabular}{rrccrcrc}
% \hline
% \multicolumn{2}{c}{Rating factors } & \multicolumn{2}{c}{Relatives} & Risk class ($i$)& $(x_{i1}, x_{i2},x_{i3})$ &Exposure & Claim count \\
% \cline{1-2}  \cline{3-4} 
%Type ($j$) &  Age ($k$) & Type ($j$) &  Age ($k$) &(Tariff cell) & & in year & observed\\
%\hline \hline
%$j=$1 & $k=$1 & $f_{11}$ & $f_{21}$& 1 & $(0,0,0)$ &89.1 & 9\\
%1 & 2 &2  & 208.5& 8\\
%1 & 3 & 3 & 155.2 & 6  \\
%2  & 1 & 4 & 19.3 & 1 \\
%2  & 2 & 5 & 360.4 & 13 \\
%2   & 3 & 6 & 276.7 & 6 \\ \hline
%\end{tabular}
%\end{center}
%\label{tab.tariff.2}
%\end{table}%
We can generalise this to a case where there many rating factors. Assume that there are $p$ predictors, $(x_1, \ldots ,x_p)$, after accounting for all relevant indicator variables. Then 

, and each rating factor has $s_k$ levels. 
\\

To do---
\begin{itemize}
  \item table like Table 2.5 and 2.6 of Ohlsson and Johansson book?
%  \item Notation $\mathbf{ x}_i\beta$ is not exact. Need to revise
  \item Generalise to many risk factor case. say $s$
\end{itemize}
%===================
\section{Further Reading and References}
%===================


 It can be the number of people when $\mu$ is the 

As most insurers maintain a risk classification system, individual policyholders are classified into one of the risk classes based on their rating factors. Consequently each risk class has a different number of policyholders and all policyholders in the same risk class would have the identical rating factors $\mathbf{ x}$ and yield the same expected loss count from the Poisson regression model. This motivates the need of the concept of the exposure. For example, suppose that risk class 1 has 100 policyholders whereas risk class 2 has 10,000. 

Therefore, we need It is common that the insurer
the loss datasets typically 
At the beginning of this chapter we assume that $i$ stands for 
Recall from the early chapters that when $X_1$ and $X_2$ are Poisson distributed independently with mean $\lambda_1$ and $\lambda_2$, respectively, the sum $X_1+X_2$ is again Poisson distributed with mean $\lambda_1+\lambda_1$. As an insurance example, if the $i$th risk class






denote $\mu_i$ as the expected value 

In the Poisson regression, we use
\begin{equation}
\label{ }
\log \mu_i=\mathbf{ x}_i\beta
\end{equation}
or 
\begin{equation}
\label{ }
\mu_i=e^{\mathbf{ x}_i\beta}
\end{equation}


\begin{itemize}
\item
\begin{align}
y|\mathbf{ x} = f(\mathbf{ x},\beta) + \varepsilon
\end{align}
for some linear or non-linear function $f$with  $E(\varepsilon)=0$\\
(Note that we us lower case $y$ and $\mathbf{ x}$ for GLM studies, as this could be more convenient)
  \item Equivalent to the above, the mean response is given by
\begin{align}
E(y|\mathbf{ x})=f(\mathbf{ x},\beta):=\mu
\end{align}
  \item As we like $E(y|\mathbf{ x})$ to vary by each level of the explanatory variable, we denote the mean response for each observation as
\begin{align}
  \mu_i=E(y_i|\mathbf{ x}_i)=f(\mathbf{ x}_i,\beta)\\
\end{align}
where $\mathbf{ x}_i=(x_{i1},\cdots,x_{ik})$ and $\beta=(\beta_1,\cdots,\beta_k)$.
\end{itemize}



\section{Application}

%================================
\section{}




\end{document}


